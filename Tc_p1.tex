\documentclass[12pt,a4paper]{article}
		\usepackage{amsmath}
		\usepackage{amsfonts}
		\usepackage{amssymb}
		\usepackage{pgf,tikz}
		\usepackage{mathrsfs}
		\usepackage{adjustbox}
		\usepackage{tabularx}
		\usepackage{multicol}
		\usepackage{etex}
		\usepackage{circuitikz}
		\usetikzlibrary {circuits.ee.IEC}
		\usepackage{pgf}
		\usepackage{bm}
		\usepackage{pstricks}
		\let\clipbox\relax
		\usetikzlibrary{arrows}
		\usepackage{lastpage}
		\usepackage{setspace}
		\usepackage{enumitem}
		\usepackage{graphicx} %table
		\usepackage{diagbox}
		\usepackage[left=1.5cm,right=1.5cm,top=1.5cm,bottom=1.5cm,includehead,includefoot]{geometry}
		\usepackage{xcolor}
		\usepackage{polyglossia}
		\usepackage{graphicx}
		\usepackage[most]{tcolorbox}
		\usepackage{titlesec}
		\usepackage{fancyhdr} % Mise en page, en-tête et pied de page
		\usepackage[a4,frame,center]{crop}
		\setdefaultlanguage[calendar=gregorian,numerals=maghrib]{arabic}
		\setotherlanguage{french}
		\newfontfamily\arabicfont[Script=Arabic,Scale=1]{Amiri}
		\newfontfamily\arabicfontsf[Script=Arabic,Scale=1]{Amiri}
		\newtcbtheorem[auto counter]{exercice}%
		{\textbf{تمرين}}{enhanced jigsaw,breakable,/tcb/top=4mm,
		attach boxed title to top right={xshift=-1cm,yshift=-3mm},
		fonttitle=\bfseries,varwidth boxed title=0.7\linewidth,
		colbacktitle=white!45!white,coltitle=white!10!black,colframe=white!50!black,
		interior style={top color=white!10!white,bottom color=blue!10!white},
		boxed title style={boxrule=0.3mm,colframe=white,
		borderline={0.1mm}{0mm}{blue!50!black},
		borderline={0.1mm}{0.4mm}{blue!50!black},
		interior style={top color=white!10!white,bottom color=white!10!white,
		middle color=blue!10!white},
		drop fuzzy shadow,arc=0mm},arc=0mm, colback=white!5!white,colframe=blue!50!white}{theorem}
		%Solution ==================================================
		\newtcbtheorem[]{solution}%
		{\textbf{حل التمرين}}{enhanced jigsaw,breakable,/tcb/top=4mm,before skip=1mm,after skip=1mm,
		attach boxed title to top center={xshift=0cm,yshift=-3.7mm},
		fonttitle=\bfseries,varwidth boxed title=0.7\linewidth,
		colbacktitle=white!45!white,coltitle=white!10!black,colframe=white!50!black,
		interior style={top color=white!10!white,bottom color=blue!10!white},
		boxed title style={boxrule=0.5mm,
		frame code={ \path[tcb fill frame] ([xshift=-4mm]frame.west)
		-- (frame.north west) -- (frame.north east) -- ([xshift=4mm]frame.east)
		-- (frame.south east) -- (frame.south west) -- cycle; },
		interior code={ \path[tcb fill interior] ([xshift=-2mm]interior.west)
		-- (interior.north west) -- (interior.north east)
		-- ([xshift=2mm]interior.east) -- (interior.south east) -- (interior.south west)
		-- cycle;} }
		,arc=0mm, colback=white!5!white,colframe=blue!50!white}{theorem}
		%============================================================
		\setlength{\columnseprule}{1pt}
		\def\columnseprulecolor{\color{blue}}
		\titlespacing{\section}{0pt}{0pt}{0pt}
		\titlespacing{\section}{0pt}{0pt}{0pt}
		\pagestyle{fancy}
		\cfoot{\thepage}
		%\rfoot{}
		\lfoot{السنة الدراسية :  
  2018/2019}
\lhead{مادة : الفيزياء والكيمياء\\الأستاذ :  مجيد نايت الياس  }
\rhead{الثانوية التأهيلية  : وادي الذهب  \\المستوى الدراسي  :  ج.م.ع }
\rfoot{التجاذب الكوني}
\lfoot{ج.م.ع }
 \chead{\centering سلسلة تمارين\\ 
التجاذب الكوني}
\begin{document}
{\setstretch{0.1}
\ \\\vspace{-0.7cm}
\noindent
.\dotfill
 
  %Exercice 3
\textbf{\begin{exercice}{}/
					إعط رتبة قدر الأعداد التالية :\\
					\begin{minipage}[c]{0.5\linewidth}
					\vspace{-0.3cm}
					\indent
					\textfrench{
						\begin{itemize}
							\item $\bm{ n_1=94= }\dotfill$
							\item $\bm{ n_2=0.0018= }\dotfill$
							\item $\bm{n_3=3.10^{2}= }\dotfill$
						\end{itemize}}
					\end{minipage}
					\begin{minipage}[c]{0.5\linewidth}
					\vspace{-0.3cm}
					\indent
					\textfrench{
						\begin{itemize}
							\item $\bm{n_4=8,7.10^{-3}=}\dotfill$
							\item $\bm{ n_5=7= }\dotfill$
							\item $\bm{ n_6=0= }\dotfill$
						\end{itemize}}
					\end{minipage}
					\end{exercice}}%=== source ===%
  %Exercice 2
\textbf{\begin{exercice}{}/
					عبر عن الأطوال التالية بالكتابة العلمية :\\			
					\begin{minipage}[c]{0.55\linewidth}
					\indent
					\textfrench{
						\begin{itemize}
							\item $\bm{ L_1=150\ 000\ 000\ km=} \dotfill$
							\item $\bm{ L_2=6\ 400\ km=} \dotfill$
							\item $\bm{ L_3=0.6012.10^3\ m=}\dotfill$
						\end{itemize}}
					\end{minipage}
					\begin{minipage}[c]{0.45\linewidth}
					\indent
					\textfrench{
						\begin{itemize}
							\item $\bm{L_4=45\ \mu m=} \dotfill$
							\item $\bm{L_5=0.005\ nm=}\dotfill$
							\item $\bm{L_6=0,300\ km=} \dotfill$
						\end{itemize}}
					\end{minipage}
					\end{exercice}}%=== source ===%
  %Exercice 5
\textbf{\begin{exercice}{}/
					اختر الجواب الصحيح :
					\begin{enumerate}
					\item قوة التجاذب الكوني المطبقة من طرف الأرض على القمر أكبر شدة من قوة التجاذب الكوني المطبقة من طرف القمر على الأرض:
					\begin{itemize}[label=$\square$]
					\item صحيح.
					\item خطأ.
					\end{itemize}
					\item شدة قوة التجاذب الكوني المطبقة بين جسمين تتناسب مع جذاء كتلتيهما :  
					\begin{itemize}[label=$\square$]
					\item صحيح.
					\item خطأ.
					\end{itemize}
					\item إذا تضاعفت المسافة بين جسمين نقطيين، فإن 
					$F$
					شدة قوة التجاذب الكوني المطبقة بينهما تصير :
					\begin{itemize}[label=$\square$]
					\item $2F$.
					\item $\dfrac{F}{2}$.
					\item $4F$.
					\item $\dfrac{F}{4}$.
					\end{itemize}
					\item وزن شخص على سطح القمر يختلف عن وزنه على سطح الارض لأن :
					\begin{itemize}[label=$\square$]
					\item الشخص بعيدا جدا عن الأرض.
					\item كتلة وشعاع القمر يختلفان عن كتلة وشعاع الارض.
					\item كتلة الشخص تغيرت.
					\end{itemize}
					\item بجوار الأرض، جسم لا يسقط على الأرض لأنه :
					\begin{itemize}[label=$\square$]
					\item لا يخضع لقوة التجاذب الكوني المطبقة من طرف الأرض.
					\item يخضع لقوة ثانية متعادلة مع قوة التجاذب الكوني المطبقة من طرف الأرض.
					\item في مدار حول الأرض.
					\end{itemize}
					\end{enumerate}
					\end{exercice}}%=== source ===%  
  
\end{document}