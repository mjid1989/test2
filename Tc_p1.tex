\documentclass[12pt,a4paper]{article}
		\usepackage{amsmath}
		\usepackage{amsfonts}
		\usepackage{amssymb}
		\usepackage{pgf,tikz}
		\usepackage{mathrsfs}
		\usepackage{adjustbox}
		\usepackage{tabularx}
		\usepackage{multicol}
		\usepackage{etex}
		\usepackage{circuitikz}
		\usetikzlibrary {circuits.ee.IEC}
		\usepackage{pgf}
		\usepackage{bm}
		\usepackage{pstricks}
		\let\clipbox\relax
		\usetikzlibrary{arrows}
		\usepackage{lastpage}
		\usepackage{setspace}
		\usepackage{enumitem}
		\usepackage{graphicx} %table
		\usepackage{diagbox}
		\usepackage[left=0.75cm,right=0.75cm,top=0.5cm,bottom=0.75cm,includehead,includefoot]{geometry}
		\usepackage{xcolor}
		\usepackage{polyglossia}
		\usepackage{graphicx}
		\usepackage[most]{tcolorbox}
		\usepackage{titlesec}
		\usepackage{fancyhdr} % Mise en page, en-tête et pied de page
		\usepackage[a4,frame,center]{crop}
		\setdefaultlanguage[calendar=gregorian,numerals=maghrib]{arabic}
		\setotherlanguage{french}
		\newfontfamily\arabicfont[Script=Arabic,Scale=1]{Amiri}
		\newfontfamily\arabicfontsf[Script=Arabic,Scale=1]{Amiri}
		\newtcbtheorem[auto counter]{exercice}%
		{\textbf{تمرين}}{enhanced jigsaw,breakable,fonttitle=\bfseries\upshape,before skip=1mm,after skip=1mm,/tcb/bottom= 1 mm ,/tcb/top= 1 mm ,
			arc=0mm, colback=white!5!white,colframe=black!50!black}{theorem}
			%Solution ==================================================
		\newtcbtheorem[]{solution}%
		{\textbf{حل التمرين}}{enhanced jigsaw,breakable,/tcb/top=4mm,before skip=1mm,after skip=1mm,
		attach boxed title to top center={xshift=0cm,yshift=-3.7mm},
		fonttitle=\bfseries,varwidth boxed title=0.7\linewidth,
		colbacktitle=white!45!white,coltitle=white!10!black,colframe=white!50!black,
		interior style={top color=white!10!white,bottom color=white!10!white},
		boxed title style={boxrule=0.5mm,
		frame code={ \path[tcb fill frame] ([xshift=-4mm]frame.west)
		-- (frame.north west) -- (frame.north east) -- ([xshift=4mm]frame.east)
		-- (frame.south east) -- (frame.south west) -- cycle; },
		interior code={ \path[tcb fill interior] ([xshift=-2mm]interior.west)
		-- (interior.north west) -- (interior.north east)
		-- ([xshift=2mm]interior.east) -- (interior.south east) -- (interior.south west)
		-- cycle;} }
		,arc=0mm, colback=white!5!white,colframe=blue!50!white}{theorem}
		%============================================================
		\setlength{\columnseprule}{1pt}
		\def\columnseprulecolor{\color{blue}}
		\titlespacing{\section}{0pt}{0pt}{0pt}
		\pagestyle{fancy}
		\cfoot{\thepage}
		%\rfoot{}
		\definecolor{color1}{RGB}{0,0,0}
		\newcommand*\circled[1]{\tikz[baseline=(char.base)]{%
        \node[shape=circle,left color=color1!60!black,right color=color1!60!black,
		middle color=color1!80!black,draw,inner sep=1pt] (char) {#1};}}
		%==============================
		\newcommand*\rectled[1]{\tikz[baseline=(char.base)]{%
        \node[shape=rectangle,left color=color1!60!black,right color=color1!60!black,
		middle color=color1!80!black,draw,inner sep=1pt] (char) {#1};}}
		\lfoot{السنة الدراسية : 
  2019/2020}
\lhead{مادة : الفيزياء والكيمياء\\الأستاذ :  اسماعيل العباسي }
\rhead{الثانوية التأهيلية  : ثانوية الروحا التاهيلية \\المستوى الدراسي  :  جذع مشترك علوم }
\rfoot{التجاذب الكوني}
\lfoot{جذع مشترك علوم }
 \chead{\centering سلسلة تمارين\\ 
التجاذب الكوني}
\begin{document}
  
  %Exercice 5
\textbf{\begin{exercice}{}/
					اختر الجواب الصحيح :
					\begin{enumerate}[label=\protect\circled{\color{white}\textbf{\arabic*}}]
					\item قوة التجاذب الكوني المطبقة من طرف الأرض على القمر أكبر شدة من قوة التجاذب الكوني المطبقة من طرف القمر على الأرض:
					\begin{itemize}[label=$\square$]
					\item صحيح.
					\item خطأ.
					\end{itemize}
					\item شدة قوة التجاذب الكوني المطبقة بين جسمين تتناسب مع جذاء كتلتيهما :  
					\begin{itemize}[label=$\square$]
					\item صحيح.
					\item خطأ.
					\end{itemize}
					\item إذا تضاعفت المسافة بين جسمين نقطيين، فإن 
					$F$
					شدة قوة التجاذب الكوني المطبقة بينهما تصير :
					\begin{itemize}[label=$\square$]
					\item $2F$.
					\item $\dfrac{F}{2}$.
					\item $4F$.
					\item $\dfrac{F}{4}$.
					\end{itemize}
					\item وزن شخص على سطح القمر يختلف عن وزنه على سطح الارض لأن :
					\begin{itemize}[label=$\square$]
					\item الشخص بعيدا جدا عن الأرض.
					\item كتلة وشعاع القمر يختلفان عن كتلة وشعاع الارض.
					\item كتلة الشخص تغيرت.
					\end{itemize}
					\item بجوار الأرض، جسم لا يسقط على الأرض لأنه :
					\begin{itemize}[label=$\square$]
					\item لا يخضع لقوة التجاذب الكوني المطبقة من طرف الأرض.
					\item يخضع لقوة ثانية متعادلة مع قوة التجاذب الكوني المطبقة من طرف الأرض.
					\item في مدار حول الأرض.
					\end{itemize}
					\end{enumerate}
					\end{exercice}}%=== source ===%
  %Exercice 2
\textbf{\begin{exercice}{}/
					عبر عن الأطوال التالية بالكتابة العلمية :\\			
					\begin{minipage}[c]{0.55\linewidth}
					\indent
					\textfrench{
						\begin{itemize}
							\item $\bm{ L_1=150\ 000\ 000\ km=} \dotfill$
							\item $\bm{ L_2=6\ 400\ km=} \dotfill$
							\item $\bm{ L_3=0.6012.10^3\ m=}\dotfill$
						\end{itemize}}
					\end{minipage}
					\begin{minipage}[c]{0.45\linewidth}
					\indent
					\textfrench{
						\begin{itemize}
							\item $\bm{L_4=45\ \mu m=} \dotfill$
							\item $\bm{L_5=0.005\ nm=}\dotfill$
							\item $\bm{L_6=0,300\ km=} \dotfill$
						\end{itemize}}
					\end{minipage}
					\end{exercice}}%=== source ===%
  %Exercice 14
\begin{exercice}{}/
					كتلة قمر اصطناعي هي : 
					$m=800kg$،
					أحسب وزنه :
					\begin{enumerate}[label=\protect\circled{\color{white}\textbf{\arabic*}}]
					\item على سطح الأرض حيث شدة التقالة هي 
					$g_0=9,81N.kg^{-1}$.
					\item على علو 
					$h=300km$
					من سطح البحر، نعطي شعاع الأرض 
					$R=6400km$.
					\end{enumerate}
					\end{exercice}%=== source ===%  
  %Exercice 12
\textbf{\begin{exercice}{}/
					نعتبر التجاذب الكوني بين الشمس والأرض.
					\\لدينا المعطيات التالية:
					\begin{itemize}
					\item نعتبر أن للشمس والأرض تماثلا لتوزع الكتلة.
					\item كتلة الشمس\hrulefill
					$\bm{M_S=1,99.10^{30}\ kg}$
					\item كتلة الأرض\hrulefill
					$\bm{M_T=5,98.10^{24}\ kg}$
					\item المسافة المتوسطة بين مركزيهما\hrulefill
					$\bm{D=1,50.10^{8}\ km}$
					\item ثابتة التجاذب الكوني\hrulefill
					$\bm{G=6,67.10^{-11}\ N.m^2.kg^{-2}}$
					\end{itemize}
					\begin{enumerate}[label=\protect\circled{\color{white}\textbf{\arabic*}}]
					\item عبر حرفيا عن شدة قوة التجاذب التي تطبقها الشمس على الأرض ثم أحسب قيمتها.
					\item استنتج شذة قوة التجاذب التي تطبقها الأرض على الشمس.
					\item  مثل على تبيانة، الشمس والأرض ومتجهة كل القوتين باستعمال السلم 
					$\bm{1\ cm \longleftrightarrow 1,00.10^{22}\ N}$.
					\end{enumerate}
					\end{exercice}}%=== source ===% 
  
\end{document}