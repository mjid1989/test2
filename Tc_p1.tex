\documentclass[12pt,a4paper]{article}
		\usepackage{amsmath}
		\usepackage{amsfonts}
		\usepackage{amssymb}
		\usepackage{pgf,tikz}
		\usepackage{mathrsfs}
		\usepackage{adjustbox}
		\usepackage{tabularx}
		\usepackage{multicol}
		\usepackage{etex}
		\usepackage{circuitikz}
		\usetikzlibrary {circuits.ee.IEC}
		\usepackage{pgf}
		\usepackage{bm}
		\usepackage{pstricks}
		\let\clipbox\relax
		\usetikzlibrary{arrows}
		\usepackage{lastpage}
		\usepackage{setspace}
		\usepackage{enumitem}
		\usepackage{graphicx} %table
		\usepackage{diagbox}
		\usepackage[left=0.75cm,right=0.75cm,top=0.5cm,bottom=0.75cm,includehead,includefoot]{geometry}
		\usepackage{xcolor}
		\usepackage{polyglossia}
		\usepackage{graphicx}
		\usepackage[most]{tcolorbox}
		\usepackage{titlesec}
		\usepackage{fancyhdr} % Mise en page, en-tête et pied de page
		\usepackage[a4,frame,center]{crop}
		\setdefaultlanguage[calendar=gregorian,numerals=maghrib]{arabic}
		\setotherlanguage{french}
		\newfontfamily\arabicfont[Script=Arabic,Scale=1]{Amiri}
		\newfontfamily\arabicfontsf[Script=Arabic,Scale=1]{Amiri}
		\newtcbtheorem[auto counter]{exercice}%
		{\textbf{تمرين}}{enhanced jigsaw,breakable,fonttitle=\bfseries\upshape,before skip=1mm,after skip=1mm,/tcb/bottom= 1 mm ,/tcb/top= 1 mm ,
			arc=0mm, colback=white!5!white,colframe=black!50!black}{theorem}
			%Solution ==================================================
		\newtcbtheorem[]{solution}%
		{\textbf{حل التمرين}}{enhanced jigsaw,breakable,/tcb/top=4mm,before skip=1mm,after skip=1mm,
		attach boxed title to top center={xshift=0cm,yshift=-3.7mm},
		fonttitle=\bfseries,varwidth boxed title=0.7\linewidth,
		colbacktitle=white!45!white,coltitle=white!10!black,colframe=white!50!black,
		interior style={top color=white!10!white,bottom color=white!10!white},
		boxed title style={boxrule=0.5mm,
		frame code={ \path[tcb fill frame] ([xshift=-4mm]frame.west)
		-- (frame.north west) -- (frame.north east) -- ([xshift=4mm]frame.east)
		-- (frame.south east) -- (frame.south west) -- cycle; },
		interior code={ \path[tcb fill interior] ([xshift=-2mm]interior.west)
		-- (interior.north west) -- (interior.north east)
		-- ([xshift=2mm]interior.east) -- (interior.south east) -- (interior.south west)
		-- cycle;} }
		,arc=0mm, colback=white!5!white,colframe=blue!50!white}{theorem}
		%============================================================
		\setlength{\columnseprule}{1pt}
		\def\columnseprulecolor{\color{blue}}
		\titlespacing{\section}{0pt}{0pt}{0pt}
		\pagestyle{fancy}
		\cfoot{\thepage}
		%\rfoot{}
		\definecolor{color1}{RGB}{0,0,0}
		\newcommand*\circled[1]{\tikz[baseline=(char.base)]{%
        \node[shape=circle,left color=color1!60!black,right color=color1!60!black,
		middle color=color1!80!black,draw,inner sep=1pt] (char) {#1};}}
		%==============================
		\newcommand*\rectled[1]{\tikz[baseline=(char.base)]{%
        \node[shape=rectangle,left color=color1!60!black,right color=color1!60!black,
		middle color=color1!80!black,draw,inner sep=1pt] (char) {#1};}}
		\lfoot{السنة الدراسية : 
  }
\lhead{مادة : الفيزياء والكيمياء\\الأستاذ :  }
\rhead{الثانوية التأهيلية  : \\المستوى الدراسي  :  }
\rfoot{التجاذب الكوني}
\lfoot{}
 \chead{\centering سلسلة تمارين\\ 
التجاذب الكوني}
\begin{document}
  
  %Exercice 2
\textbf{\begin{exercice}{}/
					عبر عن الأطوال التالية بالكتابة العلمية :\\			
					\begin{minipage}[c]{0.55\linewidth}
					\indent
					\textfrench{
						\begin{itemize}
							\item $\bm{ L_1=150\ 000\ 000\ km=} \dotfill$
							\item $\bm{ L_2=6\ 400\ km=} \dotfill$
							\item $\bm{ L_3=0.6012.10^3\ m=}\dotfill$
						\end{itemize}}
					\end{minipage}
					\begin{minipage}[c]{0.45\linewidth}
					\indent
					\textfrench{
						\begin{itemize}
							\item $\bm{L_4=45\ \mu m=} \dotfill$
							\item $\bm{L_5=0.005\ nm=}\dotfill$
							\item $\bm{L_6=0,300\ km=} \dotfill$
						\end{itemize}}
					\end{minipage}
					\end{exercice}}%=== source ===%
  %Exercice 3
\textbf{\begin{exercice}{}/
					كرتان مماثلتان ذات الكتلة 
					$\bm{m_A=m_B=100\ kg}$
					تفصل بينهما مسافة 
					$\bm{d=1m}$
					توجدان على سطح الأرض.
\begin{minipage}[c]{0.59\linewidth}
\indent					
					\begin{enumerate}[label=\protect\circled{\color{white}\textbf{\arabic*}}]
					\item أحسب شدة قوة التجاذب الكوني التي تسلطها الكرة على الأخرى.
					\item أحسب شدة قوة التجاذب الكوني التي تسلطها الأرض على إحدى الكرتين.
					\item  قارن بين هاتين الشدتين ماذا تستنتج؟
					\end{enumerate}
					\vspace{0.3cm}
\end{minipage}
					\begin{minipage}[c]{0.39\linewidth}
					\definecolor{cffffff}{RGB}{255,255,255}
\definecolor{cee7c31}{RGB}{238,124,49}
\definecolor{c3498db}{RGB}{52,152,219}
\begin{flushleft}
\begin{adjustbox}{width=\linewidth}
\fbox{
\begin{tikzpicture}[y=0.80pt, x=0.80pt, yscale=-1.000000, xscale=1.000000, inner sep=0pt, outer sep=0pt]
\begin{scope}[shift={(-334.0,-793.5)}]
  \path[shift={(335.0,798.0)},draw=black,fill=cffffff] (0.0000,0.0000) --
    (397.0000,0.0000) -- (397.0000,156.0000) -- (0.0000,156.0000) --
    (0.0000,0.0000) -- cycle;
  \path[shift={(348.61,833.0)},draw=black,fill=cffffff,line width=1.600pt]
    (0.0000,50.0000) .. controls (0.0000,22.4000) and (22.4000,0.0000) ..
    (50.0000,0.0000) .. controls (77.6000,0.0000) and (100.0000,22.4000) ..
    (100.0000,50.0000) .. controls (100.0000,77.6000) and (77.6000,100.0000) ..
    (50.0000,100.0000) .. controls (22.4000,100.0000) and (0.0000,77.6000) ..
    (0.0000,50.0000) -- cycle;
  \begin{scope}[shift={(350.0,933.0)}]
    \path[draw=cffffff,fill=cee7c31,line width=1.600pt] (0.0000,0.0000) --
      (363.0000,0.0000) -- (363.0000,12.0000) -- (0.0000,12.0000) -- (0.0000,0.0000)
      -- cycle;
    \path[shift={(2.14,0)},draw=black,line width=1.600pt] (0.0000,0.0000) --
      (360.9000,0.0000);
  \end{scope}
  \path[shift={(616.0,833.0)},draw=black,fill=cffffff,line width=1.600pt]
    (0.0000,50.0000) .. controls (0.0000,22.4000) and (22.4000,0.0000) ..
    (50.0000,0.0000) .. controls (77.6000,0.0000) and (100.0000,22.4000) ..
    (100.0000,50.0000) .. controls (100.0000,77.6000) and (77.6000,100.0000) ..
    (50.0000,100.0000) .. controls (22.4000,100.0000) and (0.0000,77.6000) ..
    (0.0000,50.0000) -- cycle;
  \begin{scope}[shift={(397.0,883.0)}]
    \path[draw=black,dash pattern=on 4.80pt off 3.20pt,line width=1.600pt]
      (2.0000,0.0000) -- (267.0000,0.0000);
    \path[draw=black,line join=round,line cap=round,line width=1.600pt]
      (8.5000,4.2000) -- (0.0000,0.0000) -- (8.5000,-4.2000);
    \path[draw=black,line join=round,line cap=round,line width=1.600pt]
      (260.5000,-4.2000) -- (269.0000,0.0000) -- (260.5000,4.2000);
  \end{scope}
  \begin{scope}[shift={(628.0,803.0)}]
    \path (38,9.5) node[] (text29) {\Large \textbf{$\bm{B}$ كرة }};
  \end{scope}
  \begin{scope}[shift={(360.61,803.0)}]
    \path (38,9.5) node[] (text29) {\Large \textbf{$\bm{A}$ كرة}};
  \end{scope}
  \begin{scope}[shift={(493.5,855.0)}]
    \path (38,9.5) node[] (text29) {\Large \textbf{$\bm{d}$}};
  \end{scope}
\end{scope}
\end{tikzpicture}}
\end{adjustbox}
\end{flushleft}
					\end{minipage}
					نعطي : ثابتة التجاذب الكوني 
					$\bm{G=6,67.10^{-11}\ N.m^2.kg^{-1}}\hrulefill$
					\end{exercice}}%=== source ===%
  
\end{document}