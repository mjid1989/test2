		\documentclass[14pt,a4paper]{extarticle}
		\usepackage{amsmath}
		\usepackage{amsfonts}
		\usepackage{amssymb}
		\usepackage{pgf,tikz}
		\usepackage{mathrsfs}
		\usepackage{adjustbox}
		\usetikzlibrary {circuits.ee.IEC}
		\usetikzlibrary{arrows}
		\usepackage{tabularx}
		%\usepackage{circuitikz}
		\usepackage{lastpage}
		\usepackage{setspace}
		\usepackage{enumitem}
		\usepackage{graphicx} %table
		\usepackage[left=0.5cm,right=0.5cm,top=1cm,bottom=1cm,includehead,includefoot]{geometry}
		\usepackage{xcolor}
		%\usepackage{fontspec}
		%\defaultfontfeatures{Mapping=tex-text}
		%\usepackage{xunicode}
		%\usepackage{xltxtra}
		%\setmainfont{???}
		\usepackage{polyglossia}
		\usepackage{graphicx}
		\usepackage[most]{tcolorbox}
		\usepackage{titlesec}
		\usepackage{fancyhdr} % Mise en page, en-tête et pied de page
		\usepackage[a4,frame,center]{crop}
		\setdefaultlanguage[calendar=gregorian,numerals=maghrib]{arabic}
		\setotherlanguage{french}
		\newfontfamily\arabicfont[Script=Arabic,Scale=1]{Arial}
		\newfontfamily\arabicfontsf[Script=Arabic,Scale=1]{Arial}
		\newtcbtheorem[auto counter]{exercice}%
		{\textbf{تمرين}}{enhanced jigsaw,breakable,fonttitle=\bfseries\upshape,
			arc=0mm, colback=white!5!white,colframe=blue!50!white}{theorem}
		\titlespacing{\section}{0pt}{0pt}{0pt}
		\pagestyle{fancy}
		\rfoot{صفحة 
		\pageref{LastPage}/\thepage}
		\cfoot{}
		\lfoot{السنة الدراسية :
2017/2018}
\lhead{مادة : الفيزياء والكيمياء\\الأستاذ : نايت الياس مجيد}
\rhead{الثانوية التأهيلية : وادي الذهب\\المستوى الدراسي : أ.ب.ع.تجريبية}
%\rfoot{01/11/2017}
\chead{\centering فرض محروس\\ 
رقم 1 الدورة الثانية
C
}
\renewcommand{\labelenumii}{\arabic{enumi} .\arabic{enumii}}
\renewcommand{\labelenumiii}{\alph{enumiii}}
\begin{document}
\vspace{-0.5 cm}
\begin{exercice}{(7ن)}/
\vspace{-0.2 cm}
يستعمل نترات الأمونيوم كثيرا في الفلاحة كسماد، ويتم تحضيره بتمرير غاز الأمونياك في محلول مائي لحمض النتريك حسب المعادلة التالية :
$$NH_{3(g)} + H_3O^+_{(aq)} \rightarrow NH^+_{4(aq)} + H_2O_{(l)}$$
\begin{enumerate}
\vspace{-0.2 cm}
\item عرف الحمض والقاعدة حسب برونشتد، وعرف الأومفوليت.
\vspace{-0.2 cm}
\item هل هذا التفاعل تفاعل حمض- قاعدة ؟ علل إجابتك.
\vspace{-0.2 cm}
\item ما المزدوجتان المشاركتان في هذا التفاعل؟
\vspace{-0.2 cm}
\item يحضر محلول مائي لحمض النتريك بتفاعل حمض النتريك
$HNO_3$
 مع الماء .
 \begin{enumerate}
 \item حدد المزدوجتان المتدخلتان في التفاعل.
 \item أكتب أنصاف المعادلات.
 \item إستنتج معادلة التفاعل.
 \end{enumerate}
\end{enumerate}
\end{exercice}%===  ===%
\vspace{-0.7 cm}
\begin{exercice}{(4ن)}/
\vspace{-0.3 cm}
\begin{enumerate}
\vspace{-0.2cm}
\item عرف المولد وأعط رمزه ومميزته. 
\vspace{-0.2cm}
\item نطبق بين مربطي محرككهربائي توترا كهربائيا مستمرا 
$U = 10V$
يمر فيه في النظام الدائم تيار كهربائي شدته 
$I=600\ mA$.
\begin{enumerate}
\vspace{-0.2cm}
\item  أحسب الطاقة الكهربائية المكتسبة من طرف المحرك خلال 
$40\ min$
بالواط-ساعة ثم بالجول.
\item أحسب القدرة الكهربائية المكتسبة من طرف المحرك.
\end{enumerate}
\end{enumerate}
\end{exercice}%=== مقرر مسار ===%
\vspace{-0.7cm}
\begin{exercice}{(9ن)}/
نركب على التوالي 
محركا كهربائيا 
$(E' = 5\ V;r'=30\ \Omega)$
ومولدا 
$(E = 24\ V;r=6\ \Omega)$
مدة 
$50\min$.
\begin{enumerate}
\item أعط تبيانة الدارة الكهربائية.
\item
أعط تعبير
 شدة التيار المار في الدارة 
ثم أحسب قيمتها.
\item أحسب التوتر بين مربطي المحرك الكهربائي.
\item أحسب الطاقة الكهربائية الممنوحة من طرف المولد.
\item أحسب الطاقة الكهربائية التي تحولت الى طاقة ميكانيكية من طرف المحرك .
\item أحسب الطاقة الحرارية المبددة بمفعول جول في مجموع الدارة.
\item ما قيمة مردود المحرك الكهربائي؟
\item ما قيمة مردود المولد؟
\item ما قيمة المردود الكلي للدارة؟
\end{enumerate}
\end{exercice}

















\end{document}