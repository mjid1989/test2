		\documentclass[12pt,a4paper]{article}
		\usepackage[utf8]{inputenc}
		\usepackage[T1]{fontenc}
		\usepackage{comment}
		\usepackage{multicol}
		\usepackage{adjustbox}
		\usepackage[francais]{babel}
		\usepackage{amsmath}
		\usepackage{amsfonts}
		\usepackage{amssymb}
		\usepackage{tabularx}
		\usepackage{lastpage}
		\usepackage{setspace}
		\usepackage{bm}
		\usepackage{graphicx} %table
		\usepackage[left=2cm,right=2cm,top=2cm,bottom=2cm,includehead,includefoot]{geometry}
		\usepackage{xcolor}
		\usepackage{graphicx}
		\usepackage[most]{tcolorbox}
		\usepackage{varwidth}
		\usepackage{titlesec}
		\usepackage{enumitem}
		\usepackage{fancyhdr} % Mise en page, en-tête et pied de page
		\setlist[enumerate]{itemsep=0mm}
		\renewcommand\headrulewidth{1pt}
\renewcommand\headwidth{\linewidth}
\renewcommand\footrulewidth{1pt}
\usepackage{fontspec}
\setromanfont{Times New Roman}
		\newtcbtheorem[auto counter]{exercice}%
		{\textbf{Exercice}}{enhanced jigsaw,breakable,
		before skip=2mm,after skip=2mm,
		colback=orange!20,colframe=black!50,boxrule=0.2mm,
		attach boxed title to top left={xshift=1cm,yshift*=1mm-\tcboxedtitleheight},
		varwidth boxed title*=-3cm,
		boxed title style={frame code={
		\path[fill=red!30!black]
		([yshift=-1mm,xshift=-1mm]frame.north west)
		arc[start angle=0,end angle=180,radius=1mm]
		([yshift=-1mm,xshift=1mm]frame.north east)
		arc[start angle=180,end angle=0,radius=1mm];
		\path[left color=red!60!black,right color=red!60!black,
		middle color=red!80!black]
		([xshift=-2mm]frame.north west) -- ([xshift=2mm]frame.north east)
		[rounded corners=1mm]-- ([xshift=1mm,yshift=-1mm]frame.north east)
		-- (frame.south east) -- (frame.south west)
		-- ([xshift=-1mm,yshift=-1mm]frame.north west)
		[sharp corners]-- cycle;
		},interior engine=empty,
		},
		fonttitle=\bfseries,
		}{theorem}		
				%Solution ==================================================
		\newtcolorbox{solution}%
		{enhanced jigsaw,breakable,
		colback=pink!20,colframe=black!50,boxrule=0.2mm
		}
		\titlespacing{\section}{0pt}{0pt}{0pt}
		\pagestyle{fancy}
		\cfoot{page	\thepage /\pageref{LastPage}}
		%\rfoot{}
		\lfoot{Ann\`ee Scolaire :
2018/2019}
\lhead{Mati\`ere : Physique Chimie\\Prof : Nait ILyasse Mjid }
\rhead{Lyc\'ee  : Oued eddahab\\Niveau  : TcT}
\rfoot{La gravitation univeselle}
\chead{\centering S\'erie d'exercices\\ 
La gravitation univeselle}
\newcommand*\circled[1]{\tikz[baseline=(char.base)]{%
            \node[shape=circle,left color=red!60!black,right color=red!60!black,
		middle color=red!80!black,draw,inner sep=1pt] (char) {#1};}}
\begin{document}
\renewcommand{\labelitemi}{$\bullet$}\textbf{\begin{exercice}{}/
					اختر الجواب الصحيح :
					\begin{enumerate}[label=\protect\circled{\color{white}\textbf{\arabic*}}]
					\item قوة التجاذب الكوني المطبقة من طرف الأرض على القمر أكبر شدة من قوة التجاذب الكوني المطبقة من طرف القمر على الأرض:
					\begin{itemize}[label=$\square$]
					\item صحيح.
					\item خطأ.
					\end{itemize}
					\item شدة قوة التجاذب الكوني المطبقة بين جسمين تتناسب مع جذاء كتلتيهما :  
					\begin{itemize}[label=$\square$]
					\item صحيح.
					\item خطأ.
					\end{itemize}
					\item إذا تضاعفت المسافة بين جسمين نقطيين، فإن 
					$F$
					شدة قوة التجاذب الكوني المطبقة بينهما تصير :
					\begin{itemize}[label=$\square$]
					\item $2F$.
					\item $\dfrac{F}{2}$.
					\item $4F$.
					\item $\dfrac{F}{4}$.
					\end{itemize}
					\item وزن شخص على سطح القمر يختلف عن وزنه على سطح الارض لأن :
					\begin{itemize}[label=$\square$]
					\item الشخص بعيدا جدا عن الأرض.
					\item كتلة وشعاع القمر يختلفان عن كتلة وشعاع الارض.
					\item كتلة الشخص تغيرت.
					\end{itemize}
					\item بجوار الأرض، جسم لا يسقط على الأرض لأنه :
					\begin{itemize}[label=$\square$]
					\item لا يخضع لقوة التجاذب الكوني المطبقة من طرف الأرض.
					\item يخضع لقوة ثانية متعادلة مع قوة التجاذب الكوني المطبقة من طرف الأرض.
					\item في مدار حول الأرض.
					\end{itemize}
					\end{enumerate}
					\end{exercice}}%=== source ===%    
\end{document}