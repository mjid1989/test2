\documentclass[12pt,a4paper]{article}
		\usepackage{amsmath}
		\usepackage{amsfonts}
		\usepackage{amssymb}
		\usepackage{pgf,tikz}
		\usepackage{mathrsfs}
		\usepackage{adjustbox}
		\usepackage{tabularx}
		\usepackage{multicol}
		\usepackage{etex}
		\usepackage{circuitikz}
		\usetikzlibrary {circuits.ee.IEC}
		\usepackage{pgf}
		\usepackage{bm}
		\usepackage{pstricks}
		\let\clipbox\relax
		\usetikzlibrary{arrows}
		\usepackage{lastpage}
		\usepackage{setspace}
		\usepackage{enumitem}
		\usepackage{graphicx} %table
		\usepackage{diagbox}
		\usepackage[left=0.75cm,right=0.75cm,top=0.5cm,bottom=0.75cm,includehead,includefoot]{geometry}
		\usepackage{xcolor}
		\usepackage{polyglossia}
		\usepackage{graphicx}
		\usepackage[most]{tcolorbox}
		\usepackage{titlesec}
		\usepackage{fancyhdr} % Mise en page, en-tête et pied de page
		\usepackage[a4,frame,center]{crop}
		\setdefaultlanguage[calendar=gregorian,numerals=maghrib]{arabic}
		\setotherlanguage{french}
		\newfontfamily\arabicfont[Script=Arabic,Scale=1]{Amiri}
		\newfontfamily\arabicfontsf[Script=Arabic,Scale=1]{Amiri}
		\newtcbtheorem[auto counter]{exercice}%
		{\textbf{تمرين}}{enhanced jigsaw,breakable,fonttitle=\bfseries\upshape,before skip=1mm,after skip=1mm,/tcb/bottom= 1 mm ,/tcb/top= 1 mm ,
			arc=0mm, colback=white!5!white,colframe=black!50!black}{theorem}
			%Solution ==================================================
		\newtcbtheorem[]{solution}%
		{\textbf{حل التمرين}}{enhanced jigsaw,breakable,/tcb/top=4mm,before skip=1mm,after skip=1mm,
		attach boxed title to top center={xshift=0cm,yshift=-3.7mm},
		fonttitle=\bfseries,varwidth boxed title=0.7\linewidth,
		colbacktitle=white!45!white,coltitle=white!10!black,colframe=white!50!black,
		interior style={top color=white!10!white,bottom color=white!10!white},
		boxed title style={boxrule=0.5mm,
		frame code={ \path[tcb fill frame] ([xshift=-4mm]frame.west)
		-- (frame.north west) -- (frame.north east) -- ([xshift=4mm]frame.east)
		-- (frame.south east) -- (frame.south west) -- cycle; },
		interior code={ \path[tcb fill interior] ([xshift=-2mm]interior.west)
		-- (interior.north west) -- (interior.north east)
		-- ([xshift=2mm]interior.east) -- (interior.south east) -- (interior.south west)
		-- cycle;} }
		,arc=0mm, colback=white!5!white,colframe=blue!50!white}{theorem}
		%============================================================
		\setlength{\columnseprule}{1pt}
		\def\columnseprulecolor{\color{blue}}
		\titlespacing{\section}{0pt}{0pt}{0pt}
		\pagestyle{fancy}
		\cfoot{\thepage}
		%\rfoot{}
		\definecolor{color1}{RGB}{0,0,0}
		\newcommand*\circled[1]{\tikz[baseline=(char.base)]{%
        \node[shape=circle,left color=color1!60!black,right color=color1!60!black,
		middle color=color1!80!black,draw,inner sep=1pt] (char) {#1};}}
		%==============================
		\newcommand*\rectled[1]{\tikz[baseline=(char.base)]{%
        \node[shape=rectangle,left color=color1!60!black,right color=color1!60!black,
		middle color=color1!80!black,draw,inner sep=1pt] (char) {#1};}}
		\lfoot{السنة الدراسية : 
  }
\lhead{مادة : الفيزياء والكيمياء\\الأستاذ :  }
\rhead{الثانوية التأهيلية  : \\المستوى الدراسي  :  }
\rfoot{قياس المواصلة  }
\lfoot{}
 \chead{\centering سلسلة تمارين\\ 
قياس المواصلة  }
\begin{document}
  
  %Exercice 1
					\textbf{\begin{exercice}{}/
أنجز عملية التحويل الى الوحدة المطلوبة في كل من الحالات التالية :
\begin{enumerate}[label=\protect\circled{\color{white}\textbf{\arabic*}}]
\vspace{-0.3cm}
\item الموصلية :
\vspace{-0.2cm}
\textfrench{
\begin{itemize}
\item $\bm{{\sigma = 65,4\ mS.m^{-1} =}}$ \dotfill $\bm{{\ S.m^{-1}}}$
\item $\bm{{\sigma = 26,7.10^{-2}\ S.cm^{-1} = }}$ \dotfill $\bm{{\  S.m^{-1}}}$
\item $\bm{{\sigma = 65,4.10^{-1}\ \mu S.cm^{-1} = }}$ \dotfill $\bm{{\  S.m^{-1}}}$
\vspace{-1cm}
\end{itemize}}
\item الموصلية المولية الأيونية :
\vspace{-0.2cm}
\textfrench{
\begin{itemize}
\item $\bm{{\lambda = 78,5.10^{-2}\ S.m^{2}.mol^{-1} = }}$ \dotfill $\bm{{\  mS.m^{2}.mol^{-1}}}$
\item $\bm{{\lambda = 5,7.10^{3}\ \mu S.m^{2}.mol^{-1} = }}$ \dotfill $\bm{{\  S.m^{2}.mol^{-1}}}$
\item $\bm{{\lambda = 19,5\ mS.m^{2}.mol^{-1} = }}$ \dotfill $\bm{{\  S.m^{2}.mol^{-1}}}$
\end{itemize}
\vspace{-1cm}
}
\item التركيز المولي :
\vspace{-0.2cm}
\textfrench{
\begin{itemize}
\item $\bm{{C = 1,3.10^{-2}\ mol.L^{-1} = }}$ \dotfill $\bm{{\  mol.m^{-3}}}$
\item $\bm{{C = 78,5\ mmol.m^{-3} = }}$ \dotfill $\bm{{\  mol.L^{-1}}}$
\item $\bm{{C = 6,5.10^{-2}\ mmol.dm^{-3} = }}$ \dotfill $\bm{{\  mol.m^{-3}}}$
\end{itemize}}
\end{enumerate}
					\end{exercice}}%===  ===%  
  %Exercice 2
					\textbf{\begin{exercice}{}/
تتكون خلية مواصلة من إلكترودين مساحة كل منهما 
$S = 2\ cm^{2}$،
تفصل بينهما مسافة 
$L = 1\ cm$.
\begin{enumerate}[label=\protect\circled{\color{white}\textbf{\arabic*}}]
\item أحسب بالوحدة العالمية النسبة 
$\dfrac{S}{L}$.
\item أعطى قياس مواصلة جزء من المحلول القيمة 
$G = 795\ \mu S$.
أحسب موصلية المحلول.
\item نحتفظ بنفس المحلول ونغير المسافة بين الإلكترودين فتأخذ 
$L$
القيمة 
$L = 2\ cm$.
\begin{enumerate}[label=\protect\rectled{\color{white}\textbf{(\alph*)}}]
 \item ما المقدار الذي تغير الواصلة 
 $G$
 أم الموصلية
 $\sigma$؟
 \item ما القيمة الجديدة للمقدار المتغير؟ 
 \end{enumerate} 
\end{enumerate}
					\end{exercice}}%===  ===% 
  
\end{document}