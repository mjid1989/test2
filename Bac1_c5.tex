\documentclass[12pt,a4paper]{article}
		\usepackage{amsmath}
		\usepackage{amsfonts}
		\usepackage{amssymb}
		\usepackage{pgf,tikz}
		\usepackage{mathrsfs}
		\usepackage{adjustbox}
		\usepackage{tabularx}
		\usepackage{multicol}
		\usepackage{etex}
		\usepackage{circuitikz}
		\usetikzlibrary {circuits.ee.IEC}
		\usepackage{pgf}
		\usepackage{bm}
		\usepackage{pstricks}
		\let\clipbox\relax
		\usetikzlibrary{arrows}
		\usepackage{lastpage}
		\usepackage{setspace}
		\usepackage{enumitem}
		\usepackage{graphicx} %table
		\usepackage{diagbox}
		\usepackage[left=0.75cm,right=0.75cm,top=0.5cm,bottom=0.75cm,includehead,includefoot]{geometry}
		\usepackage{xcolor}
		\usepackage{polyglossia}
		\usepackage{graphicx}
		\usepackage[most]{tcolorbox}
		\usepackage{titlesec}
		\usepackage{fancyhdr} % Mise en page, en-tête et pied de page
		\usepackage[a4,frame,center]{crop}
		\setdefaultlanguage[calendar=gregorian,numerals=maghrib]{arabic}
		\setotherlanguage{french}
		\newfontfamily\arabicfont[Script=Arabic,Scale=1]{Amiri}
		\newfontfamily\arabicfontsf[Script=Arabic,Scale=1]{Amiri}
		\newtcbtheorem[auto counter]{exercice}%
		{\textbf{تمرين}}{enhanced jigsaw,breakable,fonttitle=\bfseries\upshape,before skip=1mm,after skip=1mm,/tcb/bottom= 1 mm ,/tcb/top= 1 mm ,
			arc=0mm, colback=white!5!white,colframe=black!50!black}{theorem}
			%Solution ==================================================
		\newtcbtheorem[]{solution}%
		{\textbf{حل التمرين}}{enhanced jigsaw,breakable,/tcb/top=4mm,before skip=1mm,after skip=1mm,
		attach boxed title to top center={xshift=0cm,yshift=-3.7mm},
		fonttitle=\bfseries,varwidth boxed title=0.7\linewidth,
		colbacktitle=white!45!white,coltitle=white!10!black,colframe=white!50!black,
		interior style={top color=white!10!white,bottom color=white!10!white},
		boxed title style={boxrule=0.5mm,
		frame code={ \path[tcb fill frame] ([xshift=-4mm]frame.west)
		-- (frame.north west) -- (frame.north east) -- ([xshift=4mm]frame.east)
		-- (frame.south east) -- (frame.south west) -- cycle; },
		interior code={ \path[tcb fill interior] ([xshift=-2mm]interior.west)
		-- (interior.north west) -- (interior.north east)
		-- ([xshift=2mm]interior.east) -- (interior.south east) -- (interior.south west)
		-- cycle;} }
		,arc=0mm, colback=white!5!white,colframe=blue!50!white}{theorem}
		%============================================================
		\setlength{\columnseprule}{1pt}
		\def\columnseprulecolor{\color{blue}}
		\titlespacing{\section}{0pt}{0pt}{0pt}
		\pagestyle{fancy}
		\cfoot{\thepage}
		%\rfoot{}
		\definecolor{color1}{RGB}{0,0,0}
		\newcommand*\circled[1]{\tikz[baseline=(char.base)]{%
        \node[shape=circle,left color=color1!60!black,right color=color1!60!black,
		middle color=color1!80!black,draw,inner sep=1pt] (char) {#1};}}
		%==============================
		\newcommand*\rectled[1]{\tikz[baseline=(char.base)]{%
        \node[shape=rectangle,left color=color1!60!black,right color=color1!60!black,
		middle color=color1!80!black,draw,inner sep=1pt] (char) {#1};}}
		\lfoot{السنة الدراسية : 
  }
\lhead{مادة : الفيزياء والكيمياء\\الأستاذ :  }
\rhead{الثانوية التأهيلية  : \\المستوى الدراسي  :  }
\rfoot{قياس المواصلة  }
\lfoot{}
 \chead{\centering سلسلة تمارين\\ 
قياس المواصلة  }
\begin{document}
  
  %Exercice 1
					\textbf{\begin{exercice}{}/
أنجز عملية التحويل الى الوحدة المطلوبة في كل من الحالات التالية :
\begin{enumerate}[label=\protect\circled{\color{white}\textbf{\arabic*}}]
\vspace{-0.3cm}
\item الموصلية :
\vspace{-0.2cm}
\textfrench{
\begin{itemize}
\item $\bm{{\sigma = 65,4\ mS.m^{-1} =}}$ \dotfill $\bm{{\ S.m^{-1}}}$
\item $\bm{{\sigma = 26,7.10^{-2}\ S.cm^{-1} = }}$ \dotfill $\bm{{\  S.m^{-1}}}$
\item $\bm{{\sigma = 65,4.10^{-1}\ \mu S.cm^{-1} = }}$ \dotfill $\bm{{\  S.m^{-1}}}$
\vspace{-1cm}
\end{itemize}}
\item الموصلية المولية الأيونية :
\vspace{-0.2cm}
\textfrench{
\begin{itemize}
\item $\bm{{\lambda = 78,5.10^{-2}\ S.m^{2}.mol^{-1} = }}$ \dotfill $\bm{{\  mS.m^{2}.mol^{-1}}}$
\item $\bm{{\lambda = 5,7.10^{3}\ \mu S.m^{2}.mol^{-1} = }}$ \dotfill $\bm{{\  S.m^{2}.mol^{-1}}}$
\item $\bm{{\lambda = 19,5\ mS.m^{2}.mol^{-1} = }}$ \dotfill $\bm{{\  S.m^{2}.mol^{-1}}}$
\end{itemize}
\vspace{-1cm}
}
\item التركيز المولي :
\vspace{-0.2cm}
\textfrench{
\begin{itemize}
\item $\bm{{C = 1,3.10^{-2}\ mol.L^{-1} = }}$ \dotfill $\bm{{\  mol.m^{-3}}}$
\item $\bm{{C = 78,5\ mmol.m^{-3} = }}$ \dotfill $\bm{{\  mol.L^{-1}}}$
\item $\bm{{C = 6,5.10^{-2}\ mmol.dm^{-3} = }}$ \dotfill $\bm{{\  mol.m^{-3}}}$
\end{itemize}}
\end{enumerate}
					\end{exercice}}%===  ===%  
  %Exercice 2
					\textbf{\begin{exercice}{}/
تتكون خلية مواصلة من إلكترودين مساحة كل منهما 
$S = 2\ cm^{2}$،
تفصل بينهما مسافة 
$L = 1\ cm$.
\begin{enumerate}[label=\protect\circled{\color{white}\textbf{\arabic*}}]
\item أحسب بالوحدة العالمية النسبة 
$\dfrac{S}{L}$.
\item أعطى قياس مواصلة جزء من المحلول القيمة 
$G = 795\ \mu S$.
أحسب موصلية المحلول.
\item نحتفظ بنفس المحلول ونغير المسافة بين الإلكترودين فتأخذ 
$L$
القيمة 
$L = 2\ cm$.
\begin{enumerate}[label=\protect\rectled{\color{white}\textbf{(\alph*)}}]
 \item ما المقدار الذي تغير الواصلة 
 $G$
 أم الموصلية
 $\sigma$؟
 \item ما القيمة الجديدة للمقدار المتغير؟ 
 \end{enumerate} 
\end{enumerate}
					\end{exercice}}%===  ===% 
  %Exercice 3
					\textbf{\begin{exercice}{}/
يحتوي محلول مائي على أيونات الصوديوم، الألومنيوم والكلورور تراكيرها هي :
\textfrench{
\begin{itemize}
\begin{multicols}{3}
\item 
$[Na^{+}] = 2.10^{-3}\ mol.L^{-1}$
\item
$[NH_4^{+}] = 3.10^{-3}\ mol.L^{-1}$
\item
$[Cl^{-}] = 5.10^{-3}\ mol.L^{-1}$
\end{multicols}
\end{itemize}}
أحسب موصلية هذا المحلول، عند 
$25\ ^{\circ}C$.
\\\textbf{نعطي عند  
$25\ ^{\circ}C$ :
}\\
${\lambda _{Na^+} =5,0.10^{-3}\ S.m^{2}.mol^{-1}}$
و
${\lambda _{Cl^-} =7,6.10^{-3}\ S.m^{2}.mol^{-1}}$.
و
${\lambda _{NH_4^+} =7,4.10^{-3}\ S.m^{2}.mol^{-1}}$
	\end{exercice}}%===  ===% 
  %Exercice 4
					\textbf{\begin{exercice}{}/
نطبق على خلية قياس المواصلة، مغمورة كليا داخل محلول إلكتروليتي، توترا متناوبا جيبيا ذي تردد 
$500\ Hz$
وقيمته الفعالة 
$U=1,20\ V$،
فيمر تيار كهربائي شدته الفعالة 
$I=2,67\ mA$.
\begin{enumerate}[label=\protect\circled{\color{white}\textbf{\arabic*}}]
\item
 أحسب مواصلة جزء المحلول الموجود بين الإلكترودين.
\item استنتج موصلية المحلول بالوحدة 
$S.m^{-1}$.
\item بين كيف تتغير فيمتا المواصلة والموصلية إذا كانت الخلية جزئيا في المحلول.  
\end{enumerate}
نعطي المسافة بين الإلكترودين 
$L=1,50\ cm$، 
ومساحة كل إلكترود 
$120\ mm^2$.
					\end{exercice}}%=== source ===%
  %Exercice 5
					\textbf{\begin{exercice}{}/
نحصل على محلول مائي
$(S)$
لكلورور الصوديوم بإذابة كتلة
${m=11,76\ mg}$
من كلورور الصوديوم في حجم 
${V=200\ cm^3}$
من الماء.
\begin{enumerate}[label=\protect\circled{\color{white}\textbf{\arabic*}}]
\item أحسب
$C$
التركيز المولي للمحلول
$(S)$.
\item أحسب
$\sigma$
موصلية المحلول
$(S)$.
\item أحسب
$G$
مواصلة جزء المحلول
$(S)$
المغمور بين صفيحتي خلية المواصلة الذي مساحته الخارجية
$S=5\ cm^{2}$
و طوله
$L=2\ cm$.
\end{enumerate}
\textbf{نعطي:}\\
${\lambda _{Na^+} =5,0.10^{-3}\ S.m^{2}.mol^{-1}}$
و
${\lambda _{Cl^-} =7,6.10^{-3}\ S.m^{2}.mol^{-1}}$
و
${M(NaCl)=58,5\ g.mol^{-1}}$
\end{exercice}}%=== source ===% 
  %Exercice 6
					\textbf{\begin{exercice}{}/
نحضر 
$V_1=100\ mL$
من محلول مائي بإذابة 
$m=68\ mg$
من ميثانوات الصوديوم الصلب 
$HCOONa(s)$
في الماء المقطر.
\begin{enumerate}[label=\protect\circled{\color{white}\textbf{\arabic*}}]
\item أكتب معادلة الذوبان.
\item أحسب التركيز المولي 
$C$
للمذاب المستعمل.
\item إذا علمت أن ذوبان ميثانوات الصوديوم يكون كليا، اعط تراكيز الأيونات الموجودة في المحلول بالوحدة :
$mol.m^{-3}$.
\item إعط تعبير موصلية المحلول بدلالة تراكيز الأيونات الموجودة في المحلول، واحسب قيمتها.
\item نضيف كمية من الماء المقطر الى المحلول الأول ثم نقوم بقياس مواصلة جزء من المحلول الجديد باستعمال خلية ذات الخصائص التالية 
$(L=1\ cm ; S=3,21\ cm^{2})$،
نقيس قيم 
$U$
و
$I$
ونجد :
$U=1\ V$
و
$I=2,47\ mA$.
\begin{enumerate}[label=\protect\rectled{\color{white}\textbf{(\alph*)}}]
 \item أحسب المواصلة 
 $G$
 ثم استنتج موصلية المحلول الجديد.
 \item أحسب تراكيز الأيونات الموجودة في المحلول الجديد.
 \item استنتج حجم الماء المضاف الى المحلول الأول. 
 \end{enumerate} 
\end{enumerate}
\textbf{نعطي عند  
$25\ ^{\circ}C$ :
}
${\lambda _{Na^+} =5,0.10^{-3}\ S.m^{2}.mol^{-1}}$
و
${\lambda _{HCOO^-} =5,5.10^{-3}\ S.m^{2}.mol^{-1}}$\\
والكتل المولية بـ
$(g.mol^{-1})$ : 
$M(H) = 1$ ; $M(C) = 12$ ; $M(O) = 16$ ; $M(Na) = 23$
	\end{exercice}}%===  ===% 
  %Exercice 7
					\textbf{\begin{exercice}{}/
نقيس التوتر الفعال لتوتر كهربائي متناوب جيبي بين إلكتردين مغمورين في محلول ايوني وشدة
التيار الفعالة I للتيار الذي يمر في جزء من المحلول المحصول بين الإلكترودين فنجد :
$U=4,42\ V$
و
$I=2$.
\begin{enumerate}[label=\protect\circled{\color{white}\textbf{\arabic*}}]
\item أنجز تبيانة التركيب التجريبي المستعمل.
\item فسر لماذا نستعمل توترا متناوبا لقياس مواصلة محلول أيوني؟
\item أحسب مقاومة جزء المحلول المحصور بين الاكترودين.
\item إستنتج مواصلة جزء المحلول المحصور بين الالكترودين؟
\end{enumerate}
\end{exercice}}%=== source ===%  
  %Exercice 8
					\textbf{\begin{exercice}{}/
	تم تحضير محلول مخفف لحمض النتريك
$(H^+ + NO_3^-)$
تركيزه
$10^{-2}\ mol.L^{-1}$.
\begin{enumerate}[label=\protect\circled{\color{white}\textbf{\arabic*}}]
\item أحسب تراكيز مختلف الأيونات المتواجدة في المحلول بالوحدة:
$mol.m^{-3}$.
\item حدد قيمة موصلية المحلول بالوحدة
$S.m^{-1}$
ثم بالوحدة
$mS.cm^{-1}$
عند
$25^{\circ}C$.
\item أحسب المقاومية
$\rho$
للمحلول بالوحدة
$\Omega cm$.
\end{enumerate}
نعطي عند درجة الحرارة
$25^{\circ}C$ :
${\lambda _{H^+} =34,98\ mS.m^{2}.mol^{-1}}$
و
${\lambda _{NO_3^-} =7,14\ mS.m^{2}.mol^{-1}}$.	
	\end{exercice}}%=== source ===%
  
\end{document}