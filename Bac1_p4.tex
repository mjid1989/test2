\documentclass[12pt,a4paper]{article}
		\usepackage{amsmath}
		\usepackage{amsfonts}
		\usepackage{amssymb}
		\usepackage{pgf,tikz}
		\usepackage{mathrsfs}
		\usepackage{adjustbox}
		\usepackage{tabularx}
		\usepackage{multicol}
		\usepackage{etex}
		\usepackage{circuitikz}
		\usetikzlibrary {circuits.ee.IEC}
		\usepackage{pgf}
		\usepackage{bm}
		\usepackage{pstricks}
		\let\clipbox\relax
		\usetikzlibrary{arrows}
		\usepackage{lastpage}
		\usepackage{setspace}
		\usepackage{enumitem}
		\usepackage{graphicx} %table
		\usepackage{diagbox}
		\usepackage[left=0.75cm,right=0.75cm,top=0.5cm,bottom=0.75cm,includehead,includefoot]{geometry}
		\usepackage{xcolor}
		\usepackage{polyglossia}
		\usepackage{graphicx}
		\usepackage[most]{tcolorbox}
		\usepackage{titlesec}
		\usepackage{fancyhdr} % Mise en page, en-tête et pied de page
		\usepackage[a4,frame,center]{crop}
		\setdefaultlanguage[calendar=gregorian,numerals=maghrib]{arabic}
		\setotherlanguage{french}
		\newfontfamily\arabicfont[Script=Arabic,Scale=1]{Amiri}
		\newfontfamily\arabicfontsf[Script=Arabic,Scale=1]{Amiri}
		\newtcbtheorem[auto counter]{exercice}%
		{\textbf{تمرين}}{enhanced jigsaw,breakable,fonttitle=\bfseries\upshape,before skip=1mm,after skip=1mm,/tcb/bottom= 1 mm ,/tcb/top= 1 mm ,
			arc=0mm, colback=white!5!white,colframe=black!50!black}{theorem}
			%Solution ==================================================
		\newtcbtheorem[]{solution}%
		{\textbf{حل التمرين}}{enhanced jigsaw,breakable,/tcb/top=4mm,before skip=1mm,after skip=1mm,
		attach boxed title to top center={xshift=0cm,yshift=-3.7mm},
		fonttitle=\bfseries,varwidth boxed title=0.7\linewidth,
		colbacktitle=white!45!white,coltitle=white!10!black,colframe=white!50!black,
		interior style={top color=white!10!white,bottom color=white!10!white},
		boxed title style={boxrule=0.5mm,
		frame code={ \path[tcb fill frame] ([xshift=-4mm]frame.west)
		-- (frame.north west) -- (frame.north east) -- ([xshift=4mm]frame.east)
		-- (frame.south east) -- (frame.south west) -- cycle; },
		interior code={ \path[tcb fill interior] ([xshift=-2mm]interior.west)
		-- (interior.north west) -- (interior.north east)
		-- ([xshift=2mm]interior.east) -- (interior.south east) -- (interior.south west)
		-- cycle;} }
		,arc=0mm, colback=white!5!white,colframe=blue!50!white}{theorem}
		%============================================================
		\setlength{\columnseprule}{1pt}
		\def\columnseprulecolor{\color{blue}}
		\titlespacing{\section}{0pt}{0pt}{0pt}
		\pagestyle{fancy}
		\cfoot{\thepage}
		%\rfoot{}
		\definecolor{color1}{RGB}{0,0,0}
		\newcommand*\circled[1]{\tikz[baseline=(char.base)]{%
        \node[shape=circle,left color=color1!60!black,right color=color1!60!black,
		middle color=color1!80!black,draw,inner sep=1pt] (char) {#1};}}
		%==============================
		\newcommand*\rectled[1]{\tikz[baseline=(char.base)]{%
        \node[shape=rectangle,left color=color1!60!black,right color=color1!60!black,
		middle color=color1!80!black,draw,inner sep=1pt] (char) {#1};}}
		\lfoot{السنة الدراسية : 
  2018/2019}
\lhead{مادة : الفيزياء والكيمياء\\الأستاذ :  نايت الياس مجيد  }
\rhead{الثانوية التأهيلية  : وادي الذهب  \\المستوى الدراسي  :  الأولى باك ع ت  }
\rfoot{طاقة الوضع الثقالية والطاقة الميكانيكية }
\lfoot{الأولى باك ع ت  }
 \chead{\centering سلسلة تمارين\\ 
طاقة الوضع الثقالية والطاقة الميكانيكية }
\begin{document}
  
  %Exercice 1
					\begin{exercice}{}/
					نضع علبة كتلتها 
$\bm{m=1\ kg}$
فوق طاولة كما يبين الشكل جانبه :\\
\begin{minipage}{0.6\linewidth}
أحسب طاقة الوضع الثقالية للعلبة في الحالات التالية :
\begin{enumerate}[label=\protect\circled{\color{white}\textbf{\arabic*}}]
\item عند اختيار سطح الأرض كحالة مرجعية.
\item عند اختيار سطح الطاولة كحالة مرجعية.
\item عند اختيار 
$\bm{z_0 = 2\ m}$
كحالة مرجعية.
\end{enumerate}
\end{minipage}
\begin{minipage}{0.4\linewidth}
\begin{flushleft}\begin{adjustbox}{width=0.95\linewidth}
\fbox{\begin{tikzpicture}
\fill[white]  (-12,-1) rectangle (3,9);
\draw [help lines, line width = 1pt,color=orange,step=1cm] (-12,-1) grid (3,9);
\draw[->,line width = 2pt] (-10.5,0) node (v1) {} -- (-10.5,8);
\node[left] at (2,6) {\Large{\textarabic{العلبة}}};
\node[left] at (2,5) {\Large{\textarabic{الطاولة}}};
\node at (-10.8,3) {\Large{$\bm{1\ -}$}};
%\node at (-10.93,4) {\Large{$\bm{40\ -}$}};
\node at (-10.8,0) {\Large{$\bm{0\ -}$}};
\node at (-10.8,6) {\Large{$\bm{2\ -}$}};
\node at (-9.5,8) {\Large{$\bm{Z(m)}$}};
\draw  (-12,9) rectangle (3,-1);
\fill (v1) rectangle (2,-0.5);
\fill[brown] (-7,0) rectangle (0,2.5);
\fill[brown!50!black] (-7.25,3) rectangle (0.25,2.5);
\fill[gray] (-4.5,3) rectangle (-3,4);
\draw[ball color=white] (-3.75,3) circle (.1);
\draw[-,dashed,line width = 2pt] (-10.5,3) to (-3.75,3);
\draw[->,line width = 2pt] (-0.2,6) to (-3.6,4);
\draw[->,line width = 2pt] (-0.2,5) to (-2,3);
\draw[-,line width = 2pt] (-0.2,6) to (0.4,6);
\draw[-,line width = 2pt] (-0.2,5) to (0.4,5);
\end{tikzpicture}}
\end{adjustbox}
\end{flushleft}
\end{minipage}
\textbf{نعطي :}
$\bm{g=10\ N.kg^{-1}}$.
					\end{exercice}%===  ===% 
  %Exercice 2
					\begin{exercice}{}/
					نعتبر مثلثا 
$\bm{AHB}$،
قائم الزاوية في 
$\bm{H}$، حيث الضلع 
$\bm{AH}$
أفقي.
نضع 
$\bm{AB=a}$،
$\bm{\widehat{BAH} = \alpha}$.\\
ينتقل جسم نقطي كتلته 
$\bm{m}$
على الوتر 
$\bm{AB}$.
ليكن 
$\bm{M}$
موضع الجسم 
$\bm{AM=d}$.\\
\begin{minipage}{0.6\linewidth}
أعط تعبير طاقة الوضع الثقالية للجسم بدلالة 
$\bm{m}$
،
$\bm{d}$
،
$\bm{\alpha}$
،
$\bm{a}$
و
$\bm{g}$
في حالة اختيار كمرجع لطاقة الوضع الثقالية :
\begin{enumerate}[label=\protect\circled{\color{white}\textbf{\arabic*}}]
\item النقطة
$\bm{H}$.
\item النقطة
$\bm{B}$.
\item النقطة
$\bm{A}$
\end{enumerate}
\end{minipage}
\begin{minipage}{0.4\linewidth}
\begin{flushleft}
\begin{adjustbox}{width=0.8\linewidth}
\fbox{\begin{tikzpicture}
\draw [-,line width = 2pt] (2.5,6) rectangle (-7,-1.5);
\fill [white] (2.5,6) rectangle (-7,-1.5);
\draw[-,line width = 2pt] (-6,0) to (0,4);
\draw[-,line width = 2pt] (-0,0) to (0,4);
\draw[->,line width = 2pt] (1,-1) to (1,5.5);
\draw[-,line width = 2pt] (-6,0) to (0,0);
\node at (1.5,0) {\Large{$\bm{-\ O}$}};
\node at (1.5,0) {\Large{$\bm{-\ O}$}};
\node at (-6,-0.5) {\Large{$\bm{A}$}};
\node at (0,4.5) {\Large{$\bm{B}$}};
\node at (0,-0.5) {\Large{$\bm{H}$}};
\node at (1.5,5) {\Large{$\bm{z}$}};
\node at (-3,2.5) {\Large{$\bm{M}$}};
\draw[-,line width = 2pt] (-2.8,1.95) to (-3,2.2);
\draw[-,line width = 2pt] (-4.6,0) arc (0:23:2);
\draw[-,line width = 2pt] (0,0.4) to (-0.4,0.4);
\draw[-,line width = 2pt] (-0.4,0) to (-0.4,0.4);
\node at (-4,0.6) {\Large{$\bm{\alpha}$}};
\end{tikzpicture}}
\end{adjustbox}
\end{flushleft}
\end{minipage}
					\end{exercice}%===  ===%
  %Exercice 4
					\begin{exercice}{}/
					نعتبر جسم
$S$
نقطيا كتلته
$m=2\ kg$
يمكن له أن يحتل
مواضع مختلفة على المحور 
$(Oz)$
 الموجه نحو الأعلى.
 \begin{enumerate}[label=\protect\circled{\color{white}\textbf{\arabic*}}]
 \item نعتبر 
 $z_0 = 2\ m$
 أنسوب الحالة المرجعية، أحسب
  طاقة الوضع الثقالية للجسم
  $S$
  عند المواضع
  $z_1 = 6\ m$
و
	$z_2 = -4\ m$
 ثم أحسب تغير طاقة الوضع الثقالية.
\item نعتبر 
$z_0 = -1\ m$
أنسوب الحالة المرجعية،احسب
  طاقة الوضع الثقالية للجسم
  $S$
  عند المواضع
  $z_1 = 6\ m$
و
	$z_2 = -4\ m$
 ثم أحسب تغير طاقة الوضع الثقالية.ماذا تستنتج؟
 \end{enumerate}
 نعطي :
 $g=10\ N.kg^{-1}$
					\end{exercice}%===  ===% 
  %Exercice 5
					\begin{exercice}{}/
					نقذف حجرا كتلته
$m=50\ g$
 نحو الأعلى من سطح
الأرض بسرعة بدئية
$V = 72\ km.h^{-1}$.
\\نعتبر 
$z_0 = 0$
أنسوب الحالة المرجعية (سطح الأرض).
\\أحسب الارتفاع الذي يصعد إليه الحجر لكي :
\begin{enumerate}[label=\protect\circled{\color{white}\textbf{\arabic*}}]
\item  تكون طاقته الحركية مساوية لطاقة وضعه الثقالية.
\item  تكون طاقة وضعه الثقالية مساوية لطاقته الحركية البدئية.
					\end{enumerate}		
					 نعطي :
 $g=10\ N.kg^{-1}$
					\end{exercice}%===  ===% 
  %Exercice 7
					\begin{exercice}{}/
					يصعد متزلج كتلته
$m=70\ kg$
 منحدرا بسرعة ثابتة
$V=4\ m.s^{-1}$
تحت تأثير حبل.
\begin{enumerate}[label=\protect\circled{\color{white}\textbf{\arabic*}}]
\begin{minipage}{0.5\linewidth}
\item  أحسب تغير الطاقة الميكانيكية للمتزلج أثناء انتقاله من
$A$ 
إلى
$B$.
\item 
عند وصول المتزلج إلى السطح
$BC$
تحذف القوة
المطبقة من طرف الحبل، فيتوقف المتزلج عند الموضع
$C$
لينطلق بدون سرعة بدئية، فينزلق دون
احتكاك على المنحدر
$CE$
المائل بالزاوية
$\alpha_2$
بالنسبة للمستوى الأفقي المار من
$O$
الذي نتخذه كحالة
مرجعية لطاقة الوضع الثقالية.
\end{minipage}
\begin{minipage}{0.5\linewidth}
\begin{flushleft}
\begin{adjustbox}{width=0.95\linewidth}
\fbox{\begin{tikzpicture}
\fill[white]  (-12,9) rectangle (5,-1);
\draw [help lines, line width = 1pt,color=orange,step=1cm] (-12,-1) grid (5,9);
\draw[ball color=black] (-5,6) circle (.1) ;
\node at (-9.5,-0.5) {\Large{$\bm{A}$}};
\draw[ball color=black] (4,0) circle (.1) ;
\node at (-5,6.5) {\Large{$\bm{B}$}};
\draw[ball color=black] (-3,6) circle (.1) ;
\draw[ball color=black] (-9.5,0) circle (.1) ;
\draw[line width = 4pt] (4,0) -- (-9.5,0);
\node at (-3,6.5) {\Large{$\bm{C}$}};
\draw[->,line width = 2pt] (-10.5,0) -- (-10.5,8);
\node at (-11,0) {\Large{$\bm{O\ \ -}$}};
\node at (-11,1) {\Large{$\bm{20\ \ -}$}};
\node at (-10.93,2) {\Large{$\bm{40\ \ -}$}};
\node at (-10.93,3) {\Large{$\bm{60\ \ -}$}};
\node at (-10.93,4) {\Large{$\bm{80\ \ -}$}};
\node at (-10.93,5) {\Large{$\bm{100\ -}$}};
\node at (-10.93,6) {\Large{$\bm{120\ -}$}};
\node at (-9.6,7.8) {\Large{$\bm{Z(m)}$}};
\draw  (-12,9) rectangle (5,-1);
\draw[-,line width = 2pt] (-9.5,0) -- (-5,6);
\draw[-,line width = 2pt] (-3,6) -- (-5,6);
\draw[-,line width = 2pt] (4,0) -- (-3,6);
\node at (1.5,3) {\Large{$\bm{D}$}};
\draw[ball color=black] (-8.6,5.6) circle (.1) ;
\draw[->,line width = 2pt] (-8,0) arc (0:52:1.5);
\draw[->,line width = 2pt] (2.5,0) arc (-180:-220:1.5);
\node at (-7.4,0.8) {\Large{$\bm{\alpha _1}$}};
\node at (1.6,0.8) {\Large{$\bm{\alpha _2}$}};
\draw[ball color=white] (-8.8,4) circle (.25) ;
\draw[-,line width = 2pt] (-8.2,3) -- (-8.8,4);
\draw[-,line width = 2pt] (-8.2,3) -- (-7.6,2.8);
\draw[-,line width = 2pt] (-8.2,3) -- (-8,2.2);
\draw[-,line width = 2pt] (-8.4,3.4) -- (-8.2,4.2);
\draw[-,line width = 2pt] (-8.4,3.4) -- (-8.4,4.8);
\draw[-,line width = 2pt] (-8.2,4.2) -- (-8.6,5.6);
\draw[-,line width = 3pt] (-7.2,3.2) -- (-8.4,1.6);
\draw[-,line width = 3pt] (-7.2,3.2) -- (-7.2,3.6);
\draw[-,line width = 2pt] (-5.8,9) -- (-9.4,4.6);
\draw[-,line width = 2pt] (-5.8,9) -- (-9.4,4.6);
\fill (-8.8,4) circle (.25) ;
\draw[ball color=black] (1.2,2.4) circle (.1) ;
\node at (4,-0.5) {\Large{$\bm{E}$}};
\end{tikzpicture}}
\end{adjustbox}
\end{flushleft}
\end{minipage}
\begin{enumerate}
\item أحسب تغير الطاقة الميكانيكية للمتزلج بين
$B$
و
$C$
و استنتج
$Q$
كمية الحرارة الناتجة عن الاحتكاك.
\item أحسب الطاقة الميكانيكية
$E_m(C)$
للمتزلج في
الموضع
$C$.
\item أحسب الطاقة الميكانيكية
$E_m(D)$
للمتزلج في
الموضع
$D$.
\end{enumerate}
\end{enumerate}
نعطي 
$\alpha_1 = 30^o$
و 
$\alpha_2=13^o$
و
$AB = 400\ m$
و
$g=10\ N.kg^{-1}$.
					\end{exercice}%===  ===%
  %Exercice 6
					\begin{exercice}{}/
					\begin{minipage}{0.6\linewidth}
نعتبر جسما نقطيا
$(S)$
كتلته
$m=2\ kg$
ينتقل على المسار
$ABC$.
\begin{enumerate}[label=\protect\circled{\color{white}\textbf{\arabic*}}]
\item هل تنحفظ الطاقة الميكانيكية بين الموضعين
$A$
و
$C$؟
علل جوابك.
\item أحسب السرعة
$V_B$
للجسم
$(S)$
عند مروره بالموضع
$B$.
\end{enumerate}
 نعطي :
 $g=10\ N.kg^{-1}$
 ،
 $V_A=0\ m.s^{-1}$
 ،
  $V_C=20\ m.s^{-1}$.
\end{minipage}
\begin{minipage}{0.4\linewidth}
\begin{flushleft}
\begin{adjustbox}{width=0.9\linewidth}
\fbox{\begin{tikzpicture}
\fill[white]  (-12,9) rectangle (3,-1);
\draw [help lines, line width = 1pt,color=orange,step=1cm] (-12,-1) grid (3,9);
\draw[line width = 2pt] (-9.5,6) .. controls (-5,5.2) and (-6.5,0.2) .. (-2.9,0);
\draw[line width = 2pt] (2.5,1) .. controls (-0.5,8.2) and (0,0.2) .. (-2.9,0);
\draw[ball color=orange] (-5.655,3.6) circle (.5);
\draw[ball color=black] (-9.5,6) circle (.1) ;
\node at (-9.5,6.4) {\Large{$\bm{A}$}};
\draw[ball color=black] (-2.9,0) circle (.1) ;
\node at (-2.9,-0.4) {\Large{$\bm{B}$}};
\draw[ball color=black] (0.5,4) circle (.1) ;
\node at (0.5,4.4) {\Large{$\bm{C}$}};
\draw[line width = 4pt] (2.5,0) -- (-9.5,0);
\draw[<->,densely dashed,line width = 1.5pt] (-9.5,0.1) -- (-9.5,5.9);
\node at (-9,3) {\Large{$\bm{h_A}$}};
\draw[<->,densely dashed,line width = 1.5pt] (0.5,0.1) -- (0.5,3.9);
\node at (1,2) {\Large{$\bm{h_C}$}};
\node at (-4.7,4) {\Large{$\bm{(S)}$}};
\draw[->,line width = 2pt] (-10.5,0) -- (-10.5,8);
\node at (-10.93,1) {\Large{$\bm{10\ -}$}};
\node at (-10.93,2) {\Large{$\bm{20\ -}$}};
\node at (-10.93,3) {\Large{$\bm{30\ -}$}};
\node at (-10.93,4) {\Large{$\bm{40\ -}$}};
\node at (-10.93,5) {\Large{$\bm{50\ -}$}};
\node at (-10.93,6) {\Large{$\bm{60\ -}$}};
\node at (-9.6,7.8) {\Large{$\bm{h(m)}$}};
\draw  (-12,9) rectangle (3,-1);
\end{tikzpicture}}
\end{adjustbox}
\end{flushleft}
\end{minipage}
					\end{exercice}%===  ===%
  %Exercice 8
					\begin{exercice}{}/
					نعتبر كرية كتلتها
$m=41\ kg$
 في سقوط حر بدون سرعة
بدئية من موضع
$A$
$(z_A = 2\ m)$
حيث
$O$
منطبق مع
سطح الأرض الذي نعتبره كحالة مرجعية لطاقة الوضع
الثقالية.
\begin{enumerate}[label=\protect\circled{\color{white}\textbf{\arabic*}}]
\item أعط تعبير طاقة الوضع الثقالية
$E_{pp}$
للكرية عند لحظة
$t$
وحدد قيمة
$E_{pp}(A)$
لحظة بداية السقوط الحر للكرية.
\item حدد قيمة
$E_{C}(A)$
الطاقة الحركية للكرية لحظة بداية
سقوطها الحر.
\item المقدار
$E_m = E_{c}+E_{pp}$
ثابت مع مرور الزمن،
علل ذلك واحسب قيمة
$E_{m}$.
\item أوجد تعبير
$E_{c}$
بدلالة
$z$
أنسوب مركز قصور الكرية.
\item أرسم في نفس المعلم تغيرات كل من
$E_{pp}$
و
$E_{c}$
و
$E_{m}$
بدلالة
$z$
وحدد نوع التحول الطاقي الحاصل خلال
السقوط الحر للكرية.
\end{enumerate}
 نعطي :
 $g=10\ N.kg^{-1}$.
					\end{exercice}%===  ===%
  
\end{document}