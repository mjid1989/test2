\documentclass[12pt,a4paper]{article}
		\usepackage{amsmath}
		\usepackage{amsfonts}
		\usepackage{amssymb}
		\usepackage{pgf,tikz}
		\usepackage{mathrsfs}
		\usepackage{adjustbox}
		\usepackage{tabularx}
		\usepackage{multicol}
		\usepackage{etex}
		\usepackage{circuitikz}
		\usetikzlibrary {circuits.ee.IEC}
		\usepackage{pgf}
		\usepackage{bm}
		\usepackage{pstricks}
		\let\clipbox\relax
		\usetikzlibrary{arrows}
		\usepackage{lastpage}
		\usepackage{setspace}
		\usepackage{enumitem}
		\usepackage{graphicx} %table
		\usepackage{diagbox}
		\usepackage[left=1.5cm,right=1.5cm,top=1.5cm,bottom=1.5cm,includehead,includefoot]{geometry}
		\usepackage{xcolor}
		\usepackage{polyglossia}
		\usepackage{graphicx}
		\usepackage[most]{tcolorbox}
		\usepackage{varwidth}
		\usepackage{titlesec}
		\usepackage{enumitem}
		\usepackage{fancyhdr} % Mise en page, en-tête et pied de page
		\setdefaultlanguage[calendar=gregorian,numerals=maghrib]{arabic}
		\setotherlanguage{french}
		\newfontfamily\arabicfont[Script=Arabic,Scale=1]{Amiri}
		\newfontfamily\arabicfontsf[Script=Arabic,Scale=1]{Amiri}
		\author{نايت الياس}
		\setlist[enumerate]{itemsep=0mm}
		\newtcbtheorem[auto counter]{exercice}%
		{\textbf{تمرين}}{enhanced jigsaw,breakable,fonttitle=\bfseries\upshape,before skip=1mm,after skip=1mm,/tcb/bottom= 1 mm ,/tcb/top= 2 mm ,
			arc=0mm, colback=white!5!white,colframe=black,attach boxed title to top right={yshift=-3mm,xshift=-5mm},colbacktitle=white!20!white,sharpish corners,coltitle=black!40!black}{theorem}
			%Solution ==================================================
		\newtcbtheorem[]{solution}%
		{\textbf{حل التمرين}}{enhanced jigsaw,breakable,/tcb/top=4mm,before skip=1mm,after skip=1mm,
		attach boxed title to top center={xshift=0cm,yshift=-3.7mm},
		fonttitle=\bfseries,varwidth boxed title=0.7\linewidth,
		colbacktitle=white!45!white,coltitle=white!10!black,colframe=white!50!black,
		interior style={top color=white!10!white,bottom color=white!10!white},
		boxed title style={boxrule=0.5mm,
		frame code={ \path[tcb fill frame] ([xshift=-4mm]frame.west)
		-- (frame.north west) -- (frame.north east) -- ([xshift=4mm]frame.east)
		-- (frame.south east) -- (frame.south west) -- cycle; },
		interior code={ \path[tcb fill interior] ([xshift=-2mm]interior.west)
		-- (interior.north west) -- (interior.north east)
		-- ([xshift=2mm]interior.east) -- (interior.south east) -- (interior.south west)
		-- cycle;} }
		,arc=0mm, colback=white!5!white,colframe=black!50!white}{theorem}
		%============================================================
		\setlength{\columnseprule}{1pt}
		\def\columnseprulecolor{\color{black}}
		\titlespacing{\section}{0pt}{0pt}{0pt}
		\NewTColorBox[auto counter]{exercises}{+O{}}{%
		enhanced,colframe=green!20!black,colback=yellow!10!white,coltitle=green!40!black,
		fonttitle=\bfseries,
		underlay={\begin{tcbclipinterior}
		\shade[inner color=green!80!yellow,outer color=yellow!10!white]
		(interior.north west) circle (2cm);
		\draw[help lines,step=5mm,yellow!80!black,shift={(interior.north west)}]
		(interior.south west) grid (interior.north east);
		\end{tcbclipinterior}},
		title={Exercise~\thetcbcounter:},
		label={exercise@\thetcbcounter},
		attach title to upper=\quad,
		after upper={\par\hfill\textcolor{green!40!black}%
		{\itshape Solution on page~\pageref{solution@\thetcbcounter}}},
		lowerbox=ignored,
		savelowerto=solutions/exercise-\thetcbcounter.tex,
		record={\string\solution{\thetcbcounter}{solutions/exercise-\thetcbcounter.tex}},
		#1
		}		
		\titlespacing{\section}{0pt}{0pt}{0pt}
		\pagestyle{fancy}
		\cfoot{\thepage}
		%\rfoot{}
		\definecolor{color1}{RGB}{0,0,0}
		\newcommand*\circled[1]{\tikz[baseline=(char.base)]{%
        \node[shape=circle,left color=color1!60!black,right color=color1!60!black,
		middle color=color1!80!black,draw,inner sep=1pt] (char) {#1};}}
		%==============================
		\newcommand*\rectled[1]{\tikz[baseline=(char.base)]{%
        \node[shape=rectangle,left color=color1!60!black,right color=color1!60!black,
		middle color=color1!80!black,draw,inner sep=1pt] (char) {#1};}}
		\lfoot{السنة الدراسية :  
  2018/2019}
\lhead{مادة : الفيزياء والكيمياء\\الأستاذ :  بوعسرية }
\rhead{الثانوية التأهيلية  : \\المستوى الدراسي  :  الأولى باك  }
\rfoot{طاقة الوضع الثقالية والطاقة الميكانيكية }
\lfoot{الأولى باك  }
 \chead{\centering سلسلة تمارين\\ 
طاقة الوضع الثقالية والطاقة الميكانيكية }
\begin{document}
  
  %Exercice 1
					\begin{exercice}{}/
					نضع علبة كتلتها 
$\bm{m=1\ kg}$
فوق طاولة كما يبين الشكل جانبه :\\
\begin{minipage}{0.6\linewidth}
أحسب طاقة الوضع الثقالية للعلبة في الحالات التالية :
\begin{enumerate}[label=\protect\circled{\color{white}\textbf{\arabic*}}]
\item عند اختيار سطح الأرض كحالة مرجعية.
\item عند اختيار سطح الطاولة كحالة مرجعية.
\item عند اختيار 
$\bm{z_0 = 2\ m}$
كحالة مرجعية.
\end{enumerate}
\end{minipage}
\begin{minipage}{0.4\linewidth}
\begin{flushleft}\begin{adjustbox}{width=0.95\linewidth}
\fbox{\begin{tikzpicture}
\fill[white]  (-12,-1) rectangle (3,9);
\draw [help lines, line width = 1pt,color=orange,step=1cm] (-12,-1) grid (3,9);
\draw[->,line width = 2pt] (-10.5,0) node (v1) {} -- (-10.5,8);
\node[left] at (2,6) {\Large{\textarabic{العلبة}}};
\node[left] at (2,5) {\Large{\textarabic{الطاولة}}};
\node at (-10.8,3) {\Large{$\bm{1\ -}$}};
%\node at (-10.93,4) {\Large{$\bm{40\ -}$}};
\node at (-10.8,0) {\Large{$\bm{0\ -}$}};
\node at (-10.8,6) {\Large{$\bm{2\ -}$}};
\node at (-9.5,8) {\Large{$\bm{Z(m)}$}};
\draw  (-12,9) rectangle (3,-1);
\fill (v1) rectangle (2,-0.5);
\fill[brown] (-7,0) rectangle (0,2.5);
\fill[brown!50!black] (-7.25,3) rectangle (0.25,2.5);
\fill[gray] (-4.5,3) rectangle (-3,4);
\draw[ball color=white] (-3.75,3) circle (.1);
\draw[-,dashed,line width = 2pt] (-10.5,3) to (-3.75,3);
\draw[->,line width = 2pt] (-0.2,6) to (-3.6,4);
\draw[->,line width = 2pt] (-0.2,5) to (-2,3);
\draw[-,line width = 2pt] (-0.2,6) to (0.4,6);
\draw[-,line width = 2pt] (-0.2,5) to (0.4,5);
\end{tikzpicture}}
\end{adjustbox}
\end{flushleft}
\end{minipage}
\textbf{نعطي :}
$\bm{g=10\ N.kg^{-1}}$.
					\end{exercice}%===  ===% 
  %Exercice 2
					\begin{exercice}{}/
					نعتبر مثلثا 
$\bm{AHB}$،
قائم الزاوية في 
$\bm{H}$، حيث الضلع 
$\bm{AH}$
أفقي.
نضع 
$\bm{AB=a}$،
$\bm{\widehat{BAH} = \alpha}$.\\
ينتقل جسم نقطي كتلته 
$\bm{m}$
على الوتر 
$\bm{AB}$.
ليكن 
$\bm{M}$
موضع الجسم 
$\bm{AM=d}$.\\
\begin{minipage}{0.6\linewidth}
أعط تعبير طاقة الوضع الثقالية للجسم بدلالة 
$\bm{m}$
،
$\bm{d}$
،
$\bm{\alpha}$
،
$\bm{a}$
و
$\bm{g}$
في حالة اختيار كمرجع لطاقة الوضع الثقالية :
\begin{enumerate}[label=\protect\circled{\color{white}\textbf{\arabic*}}]
\item النقطة
$\bm{H}$.
\item النقطة
$\bm{B}$.
\item النقطة
$\bm{A}$
\end{enumerate}
\end{minipage}
\begin{minipage}{0.4\linewidth}
\begin{flushleft}
\begin{adjustbox}{width=0.8\linewidth}
\fbox{\begin{tikzpicture}
\draw [-,line width = 2pt] (2.5,6) rectangle (-7,-1.5);
\fill [white] (2.5,6) rectangle (-7,-1.5);
\draw[-,line width = 2pt] (-6,0) to (0,4);
\draw[-,line width = 2pt] (-0,0) to (0,4);
\draw[->,line width = 2pt] (1,-1) to (1,5.5);
\draw[-,line width = 2pt] (-6,0) to (0,0);
\node at (1.5,0) {\Large{$\bm{-\ O}$}};
\node at (1.5,0) {\Large{$\bm{-\ O}$}};
\node at (-6,-0.5) {\Large{$\bm{A}$}};
\node at (0,4.5) {\Large{$\bm{B}$}};
\node at (0,-0.5) {\Large{$\bm{H}$}};
\node at (1.5,5) {\Large{$\bm{z}$}};
\node at (-3,2.5) {\Large{$\bm{M}$}};
\draw[-,line width = 2pt] (-2.8,1.95) to (-3,2.2);
\draw[-,line width = 2pt] (-4.6,0) arc (0:23:2);
\draw[-,line width = 2pt] (0,0.4) to (-0.4,0.4);
\draw[-,line width = 2pt] (-0.4,0) to (-0.4,0.4);
\node at (-4,0.6) {\Large{$\bm{\alpha}$}};
\end{tikzpicture}}
\end{adjustbox}
\end{flushleft}
\end{minipage}
					\end{exercice}%===  ===%
  %Exercice 15
					\begin{exercice}{}/
					نعتبر كرية كتلتها
$\bm{m=200\ g}$،
معلقة إلى نقطة ثابتة
$\bm{O}$
بواسطة خيط غير
مدود، كتلته مهملة طوله
$\bm{L=80\ cm}$.
المجموعة تكون نواسا بسيطا. نمعلم
موضع النواس بالزاوية
$\bm{\theta}$
التي يقيمها الخيط مع الخط الرأسي المار من
$\bm{O}$.
نزيح النواس عن وضع توازنه نحو اليسار بالزاوية 
$\bm{\theta_1 = 30^o}$
ونرسله نحو
اليمين بالسرعة
$\bm{V_1=1,5\ m.s^{-1}}$.
\\أثناء الحركة، يبقى الخيط متمددا، نهمل الاحتكاكات ومقاومة الهواء،
ونأخذ كحالة مرجعية لطاقة الوضع الثقالية المستوى الأفقي المار من موضع
التوازن المستقر للكرية.
\begin{enumerate}[label=\protect\circled{\color{white}\textbf{\arabic*}}]
\begin{minipage}{0.6\linewidth}
\item بين أن الطاقة الميكانيكية
$\bm{E_m = E_c + E_{pp}}$
تنحفظ أثناء الحركة.أحسب
قيمتها. نأخذ
$\bm{g=10\ N.kg^{-1}}$.
\item أحسب القيمة القصوى
$\bm{\theta_m}$
التي يمكن أن يصل إليها النواس.ما
هي حركة النواس بعد ذلك؟
\item أحسب قيمة السرعة
$\bm{V_1^{'}}$
التي يجب أن تنطلق بها الكرية من
الوضعية ذات الأفصول
$\bm{\theta_1}$
لكي تتمكن من المرور فوق النقطة
$\bm{O}$
بالسرعة
${\bm{V=5\ m.s^{-1}}}$.
مع العلم أن الخيط يبقى دائما ممددا.
\end{minipage}
\begin{minipage}{0.4\linewidth}
\begin{flushleft}
\begin{adjustbox}{width=0.9\linewidth}
\fbox{\begin{tikzpicture}
\draw [-,line width = 2pt] (3,6) rectangle (-7,-2.5);
\fill [white] (3,6) rectangle (-7,-2.5);
\draw[-,line width = 2pt] (-5.6,-0.6) to (-2.4,4.8);
\draw[-,line width = 2pt] (1.6,0) to (-2.4,4.8);
\draw[->,line width = 1.5pt,dashed] (-2.4,-1.4) to (-2.4,4.8);
\draw[->,line width = 2pt] (-5.6,-0.6) to (-3.2,-1.8);
\node at (-1.2,1.6) {\Large{$\bm{\theta}$}};
\node at (-5,-1.6) {\Large{$\bm{\overrightarrow{V_1}}$}};
\node at (-2.4,5.2) {\Large{$\bm{O}$}};
\node at (-3,2.5) {\Large{$\bm{\theta_1}$}};
\draw[ball color=black] (1.6,0) circle (.5);
\draw[<-,line width = 2pt] (-3.4,3.0679) arc (-119.9993:-90:2);
\draw[->,line width = 2pt] (-2.4,1.8) arc (-90.0001:-51:3);
\draw[-,line width = 2pt,dashed] (-6.4,0) arc (-129.8056:-40:6.2482);
\draw[ball color=black] (1.6,0) circle (.5);
\end{tikzpicture}}
\end{adjustbox}
\end{flushleft}
\end{minipage}
\end{enumerate}
					\end{exercice}%===  ===% 
  %Exercice 14
					\begin{exercice}{}/
					نعتبر المدار
$\bm{ABC}$
المكون من الجزء
المستقيمي
$\bm{AB}$
والجزء الدائري
$\bm{BC}$
شعاعه
$\bm{r=OB=OC}$.
\\نرسل من النقطة
$\bm{A}$
جسما
$\bm{(S)}$
كتلته
$\bm{m=50\ g}$
شبيه بنقطة مادية بسرعة
$\bm{V_A = 5\ m.s^{-1}}$
نحو الأعلى في الاتجاه
$\bm{AB}$.\\
\begin{minipage}{0.6\linewidth}
فينزلق على الجزء
$\bm{AB}$
ليصل
إلى النقطة
$\bm{B}$
بالسرعة
$\bm{V_B}$.
\\نعتبر المستوى
$\bm{(\pi)}$
مرجعا لطاقة الوضع الثقالية .\\
نعطي :
$\bm{\alpha = 60^o}$ ; $\bm{g=10.m.s_{-2}}$ ; $\bm{AB=2\ r=1\ m}$.
\\باستعمال تغير الطاقة الميكانيكة للجسم
$\bm{(S)}$
أوجد :
\begin{enumerate}[label=\protect\circled{\color{white}\textbf{\arabic*}}]
\item تعبير السرعة
$\bm{V_B}$.
أحسب قيمتها.
\item تعبير السرعة
$\bm{V_M}$
بنقطة
$\bm{M}$
من القوس
$\bm{\widehat{BC}}$.
\\استنتج قيمة الزاوية
$\bm{\theta_{max}}$
الأفصول الزاوي للنقطة
$\bm{M}$
حيث يتوقف الجسم
$\bm{(S)}$.
\item أعط تعبير وأحسب قيمة السرعة الدنوية التي يجب أن ينطلق
بها الجسم
$\bm{(S)}$
من النقطة
$\bm{A}$
لكي يصل إلى النقطة
$\bm{C}$
في الحالتين التاليتين :
\begin{enumerate}
\item الحركة تتم بدون احتكاك طول السكة
$\bm{ABC}$.
\item الحركة تتم باحتكاك طول السكة
$\bm{ABC}$
حيث قوة الاحتكاك
$\bm{{\overrightarrow{f}}}$
تبقى مماسة للمسار وشدتها ثابتة
$\bm{f=0,2\ N}$.
\end{enumerate}
\end{enumerate}
\end{minipage}
\begin{minipage}{0.4\linewidth}
\begin{flushleft}
\begin{adjustbox}{width=0.95\linewidth}
\fbox{\begin{tikzpicture}
\fill[-,white,line width =2pt]   (3,10.5) rectangle (-6.5,-5.5);
\draw[line width =3pt] (-3.5,5) -- (-5.5,-4);
\draw[dashed,line width =3pt] (1,5) -- (-2,7.5);
\draw[dashed,line width =3pt] (-3.5,5) -- (1,5);
\draw[-,line width =3pt] (1,9) -- (1,-4);
\fill [] (2,-4) rectangle (-5.5,-4.5);
\draw [->,line width =3pt] ;
\draw[ball color=black] (2,-4) circle (.15);
\draw[ball color=black] (-5.5,-4) circle (.15);
\draw[ball color=black] (1,9) circle (.15);
\draw[ball color=black] (1,5) circle (.15);
\draw[ball color=black] (-3.5,5) circle (.15);
\node at (-6,-4) {\textbf{\Large{A}}};
\node at (1.5,5) {\textbf{\Large{O}}};
\node at (2.5,-4) {\textbf{\Large{O'}}};
\node at (1,9.5) {\textbf{\Large{C}}};
\node at (-2.5,8) {\textbf{\Large{M}}};
\node at (-6,1) {\textbf{\Large{(S)}}};
\node at (-1.5,-5) {\textbf{\Large{$(\bm{\pi})$}}};
\node at (-4,5) {\textbf{\Large{B}}};
\node at (0.5,-3.5) {\textbf{\Large{H}}};
\node at (2.5,9.5) {\textbf{\LARGE{z}}};
\node at (-1.4,5.8) {\textbf{\LARGE{$\bm{\theta}$}}};
\node at (-3,-2) {\textbf{\LARGE{$\bm{\alpha}$}}};
\draw[->,line width =3pt] ;
%\draw [-,line width =3pt](0.5,0) arc (-75.6126:200:3.9534);
\draw[<-,line width =3pt] (2,10) -- (2,-4);
\draw  [-,line width =3pt](-3.5,5) arc (168.1582:94:5);
\draw[->,line width =3pt](-1,5) arc (180:145:2);
\draw [->,line width =3pt](-3,-4) arc (0:78:2.5);
\draw[ball color=black] (-2,7.5) circle (.15);
\draw[ball color=orange] (-5,0.5) circle (.5);
\draw[ball color=black] (1,-4) circle (.15);
\end{tikzpicture}}
\end{adjustbox}
\end{flushleft}
\end{minipage}
					\end{exercice}%===  ===% 
  %Exercice 13
					\begin{exercice}{}/
					نعتبر جسما صلبا كتلته
$\bm{m=0,6\ kg}$،
قابلا للحركة على المسار
$\bm{ABCD}$
المكون من :\\
\begin{minipage}{0.5\linewidth}
\indent
\begin{itemize}
\item $AB$ : 
جزء مستقيمي طوله
$AB=3\ m$
مائل بالزاوية
$\bm{\alpha = 50^o}$
بالنسبة للمستوى الأفقي.
\item $\bm{BC}$ : 
جزء من دائرة شعاعها
$\bm{r=80\ cm}$.
\item $\bm{CD}$ : 
جزء مستقيمي أفقي طوله
$\bm{CD=3\ m}$.
\end{itemize}
\end{minipage}
\begin{minipage}{0.5\linewidth}
\begin{flushleft}
\begin{adjustbox}{width=0.95\linewidth}
\fbox{\begin{tikzpicture}
\draw[line width =2pt]   (9.5,5.5) rectangle (-9,-5);
\fill[white]   (9.5,5.5) rectangle (-9,-5);
\draw[line width =3pt] (-3.5,-2) -- (-8.5,3.5);
\draw[line width =3pt] (1,-4) arc (-90:-140:6);
\draw[line width =3pt] (1,2) -- (-3.5,-2);
\draw[dashed,line width =3pt] (-8.5,-2) -- (-3.5,-2);
\draw[<-,line width =3pt] (1,5) -- (1,-4);
\fill [] (1,-4) rectangle (9,-4.5);
\draw[ball color=orange] (-6.6,2.4) circle (.65);
\draw [->,line width =3pt](1,0.5) arc (-90.0002:-140:1.5);
\draw[ball color=white] (1,-4) circle (.15);
\draw[ball color=black] (1,-4) circle (.15);
\draw[ball color=black] (-8.5,3.5) circle (.15);
\draw[ball color=black] (9,-4) circle (.15);
\draw[ball color=black] (1,2) circle (.15);
\draw[ball color=black] (-3.5,-2) circle (.15);
\node at (-8.5,4) {\textbf{\Large{A}}};
\node at (0.5,2.5) {\textbf{\Large{O}}};
\node at (0.5,-4.5) {\textbf{\Large{C}}};
\node at (9,-3.5) {\textbf{\Large{D}}};
\node at (-5.6,3.2) {\textbf{\Large{(S)}}};
\node at (-3.5,-2.5) {\textbf{\Large{B}}};
\node at (1.5,4.5) {\textbf{\LARGE{z}}};
\node at (0,-0.5) {\textbf{\LARGE{$\bm{\alpha}$}}};
\node at (-6,-1) {\textbf{\LARGE{$\bm{\alpha}$}}};
\draw[->,line width =3pt](-5.5,-2) arc (180:132:2);
\end{tikzpicture}}
\end{adjustbox}
\end{flushleft}
\end{minipage}
\\نطلق الجسم
$\bm{(S)}$
من النقطة
$\bm{A}$
بدون سرعة بدئية،
الحركة على المسار
$\bm{ABC}$
تتم بدون احتكاك.
\\نختار المستوى الأفقي المار من
$\bm{C}$
مرجعا لطاقة الوضع الثقالية.
\\نعتبر النقطة
$\bm{C}$
أصلا للأناسيب.
\begin{enumerate}[label=\protect\circled{\color{white}\textbf{\arabic*}}]
\item عبر عن طاقة الوضع الثقالية والطاقة الميكانيكية للجسم
$\bm{(S)}$
في الموضع
$\bm{A}$
أحسب قيمها.
\item أحسب كلا من طاقة الوضع الثقالية والطاقة الحركية للجسم
$\bm{(S)}$
في الموضع
$\bm{B}$.
\item أحسب كلا من طاقة الوضع الثقالية والطاقة الحركية للجسم
$\bm{(S)}$
في الموضع
$\bm{C}$
\item يصل الجسم
$\bm{(S)}$
إلى النقطة
$\bm{D}$
بسرعة منعدمة.أحسب شدة قوة الاحتكاك بين النقطتين
$\bm{C}$
و
$\bm{D}$.
استنتج كمية الحرارة المحررة خلال الانتقال
$\bm{CD}$.
\end{enumerate}
					\end{exercice}%===  ===%
  %Exercice 12
					\begin{exercice}{}/
					نحرر جسما صلبا
$\bm{(S)}$
ذي أبعاد صغيرة جدا، كتلته
$\bm{m=100\ g}$
من نقطة
$\bm{A}$
بدون سرعة بدئية فوق مسار
نصف دائري مركزه
$\bm{O}$
وشعاعه
$\bm{R=20\ cm}$.
\\نفترض أن حركة الجسم
$\bm{(S)}$
تتم بدون احتكاك. نأخذ
المستوى الأفقي المار من النقطة
$\bm{B}$
كحالة مرجعية
لطاقة الوضع الثقالية، والنقطة
$\bm{O}$
مركز المسار
مطابقة لأصل المحور
$\bm{Oz}$ :
\begin{enumerate}[label=\protect\circled{\color{white}\textbf{\arabic*}}]
\begin{minipage}{0.6\linewidth}
\item أحسب الطاقة الميكانيكية للجسم الصلب
$\bm{(S)}$ :
\begin{enumerate}
\item عند النقطة
$\bm{A}$.
\item عند النقطة
$\bm{B}$.
\end{enumerate}
\item استنتج سرعة الجسم
$\bm{(S)}$
عند النقطة
$\bm{B}$.
\item حدد موضع النقطة
$\bm{C}$
التي يمكن للجسم
$\bm{(S)}$
أن يصعد إليها بعد تجاوزه النقطة
$\bm{B}$.
\item ما حركة
$\bm{(S)}$
بعد وصوله النقطة
$\bm{C}$؟
\end{minipage}
\begin{minipage}{0.4\linewidth}
\begin{flushleft}
\begin{adjustbox}{width=0.95\linewidth}
\fbox{\begin{tikzpicture}
\draw[line width =2pt]   (8,6.5) rectangle (-6,-5);
\fill[white]   (8,6.5) rectangle (-6,-5);
\draw[dashed,line width =3pt] (-5,2) -- (7,2);
\draw[line width =3pt] (7,2) arc (0:-180:6);
\draw[line width =3pt] (1,2) -- (-3.5,-1);
\draw[<-,line width =3pt] (1,6) -- (1,-4);
\fill [] (-5,-4) rectangle (7,-4.5);
\draw[ball color=orange] (-3.5,-1) circle (.55);
\draw [->,line width =3pt](1,0.5) arc (-90.0002:-146:1.5);
%\draw[ball color=white] (-3.5,-1) circle (.15);
\draw[ball color=black] (1,-4) circle (.15);
\draw[ball color=black] (-5,2) circle (.15);
\draw[ball color=black] (1,2) circle (.15);
\node at (-5,2.5) {\textbf{\Large{A}}};
\node at (0.5,2.5) {\textbf{\Large{O}}};
\node at (-3.5,0) {\textbf{\Large{(S)}}};
\node at (0.5,-3.5) {\textbf{\Large{B}}};
\node at (-4.5,-1.5) {\textbf{\Large{M}}};
\node at (1.5,5.5) {\textbf{\LARGE{z}}};
\node at (0,0) {\textbf{\LARGE{$\bm{\theta}$}}};
\end{tikzpicture}}
\end{adjustbox}
\end{flushleft}
\end{minipage}
\item يمكن معلمة الموضع
$\bm{M}$
للجسم
$\bm{(S)}$
بالزاوية
$\bm{\theta =\widehat{BOM}}$
أو بالأنسوب
$\bm{z}$
على المحور الرأسي
$\bm{Oz}$
الموجه نحو الأعلى. مثل مبيانيا تغيرات طاقة الوضع الثقالية
$\bm{E_{pp}}$
و الطاقة الحركية
$\bm{E_{c}}$
والطاقة الميكانيكية
$\bm{E_{m}}$
بدلالة :
\begin{enumerate}
\item الأنسوب
$\bm{z}$.
\item الزاوية
$\bm{\theta}$
\end{enumerate}
\end{enumerate}
					\end{exercice}%===  ===% 
  
\end{document}