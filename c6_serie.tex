		\documentclass[14pt,a4paper]{extarticle}
		\usepackage{amsmath}
		\usepackage{amsfonts}
		\usepackage{amssymb}
		\usepackage{pgf,tikz}
		\usepackage{mathrsfs}
		\usepackage{adjustbox}
		\usetikzlibrary {circuits.ee.IEC}
		\usetikzlibrary{arrows}
		\usepackage{tabularx}
		\usepackage{multicol}
		%\usepackage{circuitikz}
		\usepackage{lastpage}
		\usepackage{setspace}
		\usepackage{enumitem}
		\usepackage{graphicx} %table
		\usepackage[left=0.5cm,right=0.5cm,top=1cm,bottom=1cm,includehead,includefoot]{geometry}
		\usepackage{xcolor}
		%\usepackage{fontspec}
		%\defaultfontfeatures{Mapping=tex-text}
		%\usepackage{xunicode}
		%\usepackage{xltxtra}
		%\setmainfont{???}
		
		\usepackage{polyglossia}
		\usepackage{graphicx}
		\usepackage[most]{tcolorbox}
		\usepackage{titlesec}
		\usepackage{fancyhdr} % Mise en page, en-tête et pied de page
		\usepackage[a4,frame,center]{crop}
		\setdefaultlanguage[calendar=gregorian,numerals=maghrib]{arabic}
		\setotherlanguage{french}
		\newfontfamily\arabicfont[Script=Arabic,Scale=1]{Arial}
		\newfontfamily\arabicfontsf[Script=Arabic,Scale=1]{Arial}
		\newtcbtheorem[auto counter]{exercice}%
		{\textbf{تمرين}}{enhanced jigsaw,breakable,fonttitle=\bfseries\upshape,
			arc=0mm, colback=white!5!white,colframe=blue!50!white}{theorem}
		\titlespacing{\section}{0pt}{0pt}{0pt}
		\pagestyle{fancy}
		\rfoot{صفحة 
		\pageref{LastPage}/\thepage}
		%\rfoot{}
		\lfoot{السنة الدراسية :
2017/2018}
\lhead{مادة : الفيزياء والكيمياء\\الأستاذ : نايت الياس مجيد}
\rhead{ الثانوية التأهيلية : وادي الذهب\\المستوى الدراسي : أ.ب.ع.تجريبية}
\cfoot{ }
\chead{\centering سلسلة تمارين\\ 
التفاعلات الحمضية-القاعدية}
\begin{document}
{\setstretch{0.01}
\begin{exercice}{}/
أكتب نصف المعادلة لكل مزدوجة :

\begin{minipage}{0.49\linewidth}
\indent
\begin{enumerate}
\item
$HCO_2H/HCO_2^{-}$
\item
$H_2O/HO^{-}$
\item
$CH_3CO_2H/CH_3CO_2^{-}$
\item
$NH_4^{+}/NH_3$
\end{enumerate}
\end{minipage}
\begin{minipage}{0.5\linewidth}
\begin{enumerate}
\setcounter{enumi}{4}
\item
$H_3O^{+}/H_2O$
\item
$HCl/Cl^{-}$
\item
$BrOH/BrO^{-}$
\item
$HNO_3/NO_3^{-}$
\end{enumerate}
\end{minipage}
\end{exercice}{}
%Exercice -----------------------------
\begin{exercice}{}/
عين بالنسبة لكل تفاعل حمض – قاعدة المزدوجتين
قاعدة⁄حمض المتدخلتين في التفاعل.
\begin{enumerate}
\item \hrulefill
$NH_3 + CH_3CO_2H \rightarrow NH_4^{+} + CH_3CO_2$
\item \hrulefill
$H_2SO_3 + C_2H_5NH_2 \rightarrow C_2H_5NH_3^{+} + HSO_3^{-}$ 
\item \hrulefill
$CO_3^{2-} + HCO_2H \rightarrow HCO_3^{-} + HCO_2^{-}$
\item \hrulefill
$HSO_3^{-} + HO^{-} \rightarrow SO_3^{2-} + H_2O$
\item \hrulefill
$CO_2,H_2O + HO^{-} \rightarrow HCO_3^{-} + H_2O$
\end{enumerate}
\end{exercice}{}
\begin{exercice}{}/
\begin{enumerate}
\item
نضيف بعض القطرات من الكاشف الملون
$BBT$
في شكله القاعدي
$(Ind^-)$
على قليل من محلول مائي لكلورور الهيدروجين (حمض الكلوريدريك).
\begin{enumerate}
\item ما اللون الذي سيأخذه الخليط ؟
\item أكتب معادلة التفاعل الحاصل, و حدد المزدوجتين قاعدة/حمض المتفاعلتين.
\end{enumerate}
\item نضيف بعض القطرات من الكاشف الملون
$BBT$
في شكله الحمضي
$(HInd)$
على قليل من محلول مائي لهيدروكسيد الصوديوم (محلول الصودا).
\begin{enumerate}
\item ما اللون الذي سيأخذه الخليط ؟
\item أكتب معادلة التفاعل الحاصل, و حدد المزدوجتين قاعدة/حمض المتفاعلتين.
\end{enumerate}
\item ماذا يمكنك استنتاجه بالنسبة للماء؟
\end{enumerate}				
					\end{exercice}%===  ===%
\begin{exercice}{}/
يرجع التشنج العضلي عند الرياضيين إلى تكون الحمض اللبني
${C_3H_6O_3}_{(aq)}$
في العضلات.
\begin{enumerate}
\item أعط صيغة القاعدة المرافقة لهذا الحمض.
\item يتفاعل الحمض اللبني مع أيونات هيدروجينو كربونات
$HCO^-_{3(aq)}$
الموجود في الدم . اكتب معادلة التفاعل الحاصل.
\item يتفاعل الحمض اللبني كذلك مع أيونات هيدروجينو فوسفات
$HPO^{2-}_{4(aq)}$.
ما المزدوجتان المتدخلتان؟
اكتب نصفي المعادلة حمض قاعدة و استنتج المعادلة الحصيلة.
\item ما المزدوجة الثانية التي ينتمي إليها أيون
$HPO^{2-}_{4(aq)}$؟
و ما دوره فيها؟ و ماذا تستنتج؟
\end{enumerate}
					\end{exercice}%===  ===%
\begin{exercice}{}/
نعتبر التفاعل بين أيونات السيانور وأيونات الأوكسونيوم :
$$H_3O^{+}_{(aq)} +CN^{-}_{(aq)} \rightarrow HCN_{(aq)} + H_2O_{(l)}$$
\begin{enumerate}
\item عين المزدوجتين قاعدة/حمض المتفاعلتين.
\item نحضر حجما
$V_1 = 500\ mL$
 لأيونات السيانور بإذابة
$m=3,0\ g$
من سيانور البوتاسيوم في الماء الخالص.
\begin{enumerate}
\item أحسب التركيز 
$C_1$
المولي للمحلول.
\item ما الحجم
$V_2$
 اللازم استعماله من محلول حمض
الكلوريدريك ذي تركيز
$C_2 = 0,1\ mol.l^{-1}$
 لتتفاعل
الأيونات 
$CN^{-}$
كليا.
\end{enumerate}
\end{enumerate}
\end{exercice}{}
\begin{exercice}{}/
يحتوي قرص الأسبرين الفائر على حمض
أستيلسليسيليك 
$C_9H_8O_4$
وحمض السيتريك
$C_6H_8O_7$
وهيدروجينوكربونات الصوديوم
$NaHCO_3$
وجسم صلب.
\begin{enumerate}
\item أكتب نصف معادلة تفاعل الحمض
$C_9H_8O_4$.
\item أيون هيدروجينوكربونات يعتبر أمفوليتا. عرف
الأمفوليت .
\item أكتب نصفي المعادلتين اللتين يتدخل فيهما
$HCO_3^{-}$.
\item أكتب معادلة التفاعل بين الحمض
$C_9H_8O_4$
والأيون
$HCO_3^{-}$
تشير بطاقة دواء الأسبرين إلى أن كل قرص فائر
يحتوي على :
\begin{itemize}
\item $324\ mg$
من الحمض
$C_9H_8O_4$.
\item $1,625\ g$
من الحمض
$NaHCO_3$.
\item $0,965\ g$
من الحمض
$C_6H_8O_7$.
\end{itemize}
\begin{enumerate}
\item  أحسب كمية مادة أيونات
$HCO_3^{-}$
اللازمة
للتفاعل مع كل الحمض
$C_9H_8O_4$
الموجود في القرص.
\item أحسب كتلة هيدروجينوكربونات الصوديوم
الموافقة وقارنها بالكتلة الموجودة في القرص ، ماذا
تستنتج ؟
\end{enumerate}
\end{enumerate}
\textbf{نعطي :}
$M(C_9H_8O_4) = 180\ g.mol^{-1}$
و
$M(NaHCO_3) = 84\ g.mol^{-1}$
\end{exercice}{}
}
\end{document}